% Options for packages loaded elsewhere
\PassOptionsToPackage{unicode}{hyperref}
\PassOptionsToPackage{hyphens}{url}
%
\documentclass[
]{article}
\usepackage{amsmath,amssymb}
\usepackage{lmodern}
\usepackage{iftex}
\ifPDFTeX
  \usepackage[T1]{fontenc}
  \usepackage[utf8]{inputenc}
  \usepackage{textcomp} % provide euro and other symbols
\else % if luatex or xetex
  \usepackage{unicode-math}
  \defaultfontfeatures{Scale=MatchLowercase}
  \defaultfontfeatures[\rmfamily]{Ligatures=TeX,Scale=1}
\fi
% Use upquote if available, for straight quotes in verbatim environments
\IfFileExists{upquote.sty}{\usepackage{upquote}}{}
\IfFileExists{microtype.sty}{% use microtype if available
  \usepackage[]{microtype}
  \UseMicrotypeSet[protrusion]{basicmath} % disable protrusion for tt fonts
}{}
\makeatletter
\@ifundefined{KOMAClassName}{% if non-KOMA class
  \IfFileExists{parskip.sty}{%
    \usepackage{parskip}
  }{% else
    \setlength{\parindent}{0pt}
    \setlength{\parskip}{6pt plus 2pt minus 1pt}}
}{% if KOMA class
  \KOMAoptions{parskip=half}}
\makeatother
\usepackage{xcolor}
\usepackage[margin=1in]{geometry}
\usepackage{color}
\usepackage{fancyvrb}
\newcommand{\VerbBar}{|}
\newcommand{\VERB}{\Verb[commandchars=\\\{\}]}
\DefineVerbatimEnvironment{Highlighting}{Verbatim}{commandchars=\\\{\}}
% Add ',fontsize=\small' for more characters per line
\usepackage{framed}
\definecolor{shadecolor}{RGB}{248,248,248}
\newenvironment{Shaded}{\begin{snugshade}}{\end{snugshade}}
\newcommand{\AlertTok}[1]{\textcolor[rgb]{0.94,0.16,0.16}{#1}}
\newcommand{\AnnotationTok}[1]{\textcolor[rgb]{0.56,0.35,0.01}{\textbf{\textit{#1}}}}
\newcommand{\AttributeTok}[1]{\textcolor[rgb]{0.77,0.63,0.00}{#1}}
\newcommand{\BaseNTok}[1]{\textcolor[rgb]{0.00,0.00,0.81}{#1}}
\newcommand{\BuiltInTok}[1]{#1}
\newcommand{\CharTok}[1]{\textcolor[rgb]{0.31,0.60,0.02}{#1}}
\newcommand{\CommentTok}[1]{\textcolor[rgb]{0.56,0.35,0.01}{\textit{#1}}}
\newcommand{\CommentVarTok}[1]{\textcolor[rgb]{0.56,0.35,0.01}{\textbf{\textit{#1}}}}
\newcommand{\ConstantTok}[1]{\textcolor[rgb]{0.00,0.00,0.00}{#1}}
\newcommand{\ControlFlowTok}[1]{\textcolor[rgb]{0.13,0.29,0.53}{\textbf{#1}}}
\newcommand{\DataTypeTok}[1]{\textcolor[rgb]{0.13,0.29,0.53}{#1}}
\newcommand{\DecValTok}[1]{\textcolor[rgb]{0.00,0.00,0.81}{#1}}
\newcommand{\DocumentationTok}[1]{\textcolor[rgb]{0.56,0.35,0.01}{\textbf{\textit{#1}}}}
\newcommand{\ErrorTok}[1]{\textcolor[rgb]{0.64,0.00,0.00}{\textbf{#1}}}
\newcommand{\ExtensionTok}[1]{#1}
\newcommand{\FloatTok}[1]{\textcolor[rgb]{0.00,0.00,0.81}{#1}}
\newcommand{\FunctionTok}[1]{\textcolor[rgb]{0.00,0.00,0.00}{#1}}
\newcommand{\ImportTok}[1]{#1}
\newcommand{\InformationTok}[1]{\textcolor[rgb]{0.56,0.35,0.01}{\textbf{\textit{#1}}}}
\newcommand{\KeywordTok}[1]{\textcolor[rgb]{0.13,0.29,0.53}{\textbf{#1}}}
\newcommand{\NormalTok}[1]{#1}
\newcommand{\OperatorTok}[1]{\textcolor[rgb]{0.81,0.36,0.00}{\textbf{#1}}}
\newcommand{\OtherTok}[1]{\textcolor[rgb]{0.56,0.35,0.01}{#1}}
\newcommand{\PreprocessorTok}[1]{\textcolor[rgb]{0.56,0.35,0.01}{\textit{#1}}}
\newcommand{\RegionMarkerTok}[1]{#1}
\newcommand{\SpecialCharTok}[1]{\textcolor[rgb]{0.00,0.00,0.00}{#1}}
\newcommand{\SpecialStringTok}[1]{\textcolor[rgb]{0.31,0.60,0.02}{#1}}
\newcommand{\StringTok}[1]{\textcolor[rgb]{0.31,0.60,0.02}{#1}}
\newcommand{\VariableTok}[1]{\textcolor[rgb]{0.00,0.00,0.00}{#1}}
\newcommand{\VerbatimStringTok}[1]{\textcolor[rgb]{0.31,0.60,0.02}{#1}}
\newcommand{\WarningTok}[1]{\textcolor[rgb]{0.56,0.35,0.01}{\textbf{\textit{#1}}}}
\usepackage{graphicx}
\makeatletter
\def\maxwidth{\ifdim\Gin@nat@width>\linewidth\linewidth\else\Gin@nat@width\fi}
\def\maxheight{\ifdim\Gin@nat@height>\textheight\textheight\else\Gin@nat@height\fi}
\makeatother
% Scale images if necessary, so that they will not overflow the page
% margins by default, and it is still possible to overwrite the defaults
% using explicit options in \includegraphics[width, height, ...]{}
\setkeys{Gin}{width=\maxwidth,height=\maxheight,keepaspectratio}
% Set default figure placement to htbp
\makeatletter
\def\fps@figure{htbp}
\makeatother
\setlength{\emergencystretch}{3em} % prevent overfull lines
\providecommand{\tightlist}{%
  \setlength{\itemsep}{0pt}\setlength{\parskip}{0pt}}
\setcounter{secnumdepth}{5}
\ifLuaTeX
  \usepackage{selnolig}  % disable illegal ligatures
\fi
\IfFileExists{bookmark.sty}{\usepackage{bookmark}}{\usepackage{hyperref}}
\IfFileExists{xurl.sty}{\usepackage{xurl}}{} % add URL line breaks if available
\urlstyle{same} % disable monospaced font for URLs
\hypersetup{
  pdftitle={Problem},
  hidelinks,
  pdfcreator={LaTeX via pandoc}}

\title{Problem}
\author{}
\date{\vspace{-2.5em}2022-12-23}

\begin{document}
\maketitle

{
\setcounter{tocdepth}{2}
\tableofcontents
}
\hypertarget{package-overview}{%
\section{Package overview(?)}\label{package-overview}}

\hypertarget{dependencies}{%
\subsection{Dependencies}\label{dependencies}}

These would need need to be imported in finalized package

\begin{Shaded}
\begin{Highlighting}[]
\FunctionTok{library}\NormalTok{(tidyverse)}
\end{Highlighting}
\end{Shaded}

\begin{verbatim}
## -- Attaching packages --------------------------------------- tidyverse 1.3.1 --
\end{verbatim}

\begin{verbatim}
## v ggplot2 3.3.6     v purrr   0.3.4
## v tibble  3.1.7     v dplyr   1.0.9
## v tidyr   1.2.0     v stringr 1.4.0
## v readr   2.1.2     v forcats 0.5.1
\end{verbatim}

\begin{verbatim}
## -- Conflicts ------------------------------------------ tidyverse_conflicts() --
## x dplyr::filter() masks stats::filter()
## x dplyr::lag()    masks stats::lag()
\end{verbatim}

\begin{Shaded}
\begin{Highlighting}[]
\FunctionTok{library}\NormalTok{(lubridate)}
\end{Highlighting}
\end{Shaded}

\begin{verbatim}
## 
## Attaching package: 'lubridate'
\end{verbatim}

\begin{verbatim}
## The following objects are masked from 'package:base':
## 
##     date, intersect, setdiff, union
\end{verbatim}

\begin{Shaded}
\begin{Highlighting}[]
\FunctionTok{library}\NormalTok{(amt)}
\end{Highlighting}
\end{Shaded}

\begin{verbatim}
## 
## Attaching package: 'amt'
\end{verbatim}

\begin{verbatim}
## The following object is masked from 'package:stats':
## 
##     filter
\end{verbatim}

\begin{Shaded}
\begin{Highlighting}[]
\FunctionTok{library}\NormalTok{(raster)}
\end{Highlighting}
\end{Shaded}

\begin{verbatim}
## Loading required package: sp
\end{verbatim}

\begin{verbatim}
## 
## Attaching package: 'sp'
\end{verbatim}

\begin{verbatim}
## The following object is masked from 'package:amt':
## 
##     bbox
\end{verbatim}

\begin{verbatim}
## 
## Attaching package: 'raster'
\end{verbatim}

\begin{verbatim}
## The following object is masked from 'package:amt':
## 
##     select
\end{verbatim}

\begin{verbatim}
## The following object is masked from 'package:dplyr':
## 
##     select
\end{verbatim}

\begin{Shaded}
\begin{Highlighting}[]
\FunctionTok{library}\NormalTok{(parallel)}
\FunctionTok{library}\NormalTok{(data.table)}
\end{Highlighting}
\end{Shaded}

\begin{verbatim}
## 
## Attaching package: 'data.table'
\end{verbatim}

\begin{verbatim}
## The following object is masked from 'package:raster':
## 
##     shift
\end{verbatim}

\begin{verbatim}
## The following objects are masked from 'package:lubridate':
## 
##     hour, isoweek, mday, minute, month, quarter, second, wday, week,
##     yday, year
\end{verbatim}

\begin{verbatim}
## The following objects are masked from 'package:dplyr':
## 
##     between, first, last
\end{verbatim}

\begin{verbatim}
## The following object is masked from 'package:purrr':
## 
##     transpose
\end{verbatim}

\begin{Shaded}
\begin{Highlighting}[]
\FunctionTok{library}\NormalTok{(Matrix)}
\end{Highlighting}
\end{Shaded}

\begin{verbatim}
## 
## Attaching package: 'Matrix'
\end{verbatim}

\begin{verbatim}
## The following objects are masked from 'package:tidyr':
## 
##     expand, pack, unpack
\end{verbatim}

\begin{Shaded}
\begin{Highlighting}[]
\FunctionTok{library}\NormalTok{(progress)}
\FunctionTok{library}\NormalTok{(pbapply)}
\FunctionTok{library}\NormalTok{(pbmcapply)}
\FunctionTok{library}\NormalTok{(igraph)}
\end{Highlighting}
\end{Shaded}

\begin{verbatim}
## 
## Attaching package: 'igraph'
\end{verbatim}

\begin{verbatim}
## The following object is masked from 'package:raster':
## 
##     union
\end{verbatim}

\begin{verbatim}
## The following objects are masked from 'package:lubridate':
## 
##     %--%, union
\end{verbatim}

\begin{verbatim}
## The following objects are masked from 'package:dplyr':
## 
##     as_data_frame, groups, union
\end{verbatim}

\begin{verbatim}
## The following objects are masked from 'package:purrr':
## 
##     compose, simplify
\end{verbatim}

\begin{verbatim}
## The following object is masked from 'package:tidyr':
## 
##     crossing
\end{verbatim}

\begin{verbatim}
## The following object is masked from 'package:tibble':
## 
##     as_data_frame
\end{verbatim}

\begin{verbatim}
## The following objects are masked from 'package:stats':
## 
##     decompose, spectrum
\end{verbatim}

\begin{verbatim}
## The following object is masked from 'package:base':
## 
##     union
\end{verbatim}

\hypertarget{create-the-surface-prediction}{%
\subsection{Create the surface
prediction}\label{create-the-surface-prediction}}

To side-step issues where rasters are of different resolution, we can
create our own custom grid for predictions that match the most confined
extents of the rasters. We also might want to specify the resolution of
that underlying raster. If that raster is in utm (as defined below), we
can specify x and y cell sizes

\begin{Shaded}
\begin{Highlighting}[]
\NormalTok{create\_mock\_surface }\OtherTok{\textless{}{-}} \ControlFlowTok{function}\NormalTok{(raster.list, }\AttributeTok{multiple.extents =}\NormalTok{ F, }\AttributeTok{resolution =} \FunctionTok{list}\NormalTok{(}\AttributeTok{x =} \DecValTok{100}\NormalTok{, }\AttributeTok{y =} \DecValTok{100}\NormalTok{))\{}
  
  \CommentTok{\# If rasters are all of the same extent, take the extent}
  \ControlFlowTok{if}\NormalTok{(multiple.extents }\SpecialCharTok{==}\NormalTok{ F)\{}
\NormalTok{    xmin }\OtherTok{\textless{}{-}} \FunctionTok{extent}\NormalTok{(raster.list)[}\DecValTok{1}\NormalTok{]}
\NormalTok{    xmax }\OtherTok{\textless{}{-}} \FunctionTok{extent}\NormalTok{(raster.list)[}\DecValTok{2}\NormalTok{]}
\NormalTok{    ymin }\OtherTok{\textless{}{-}} \FunctionTok{extent}\NormalTok{(raster.list)[}\DecValTok{3}\NormalTok{]}
\NormalTok{    ymax }\OtherTok{\textless{}{-}} \FunctionTok{extent}\NormalTok{(raster.list)[}\DecValTok{4}\NormalTok{]}
\NormalTok{  \}}
  
  \CommentTok{\# If rasters are of different extent, take the overlap extent}
  \ControlFlowTok{if}\NormalTok{(multiple.extents }\SpecialCharTok{==}\NormalTok{ T)\{}
\NormalTok{    xmin }\OtherTok{\textless{}{-}} \FunctionTok{max}\NormalTok{(}\FunctionTok{unlist}\NormalTok{(}\FunctionTok{lapply}\NormalTok{(raster.list, }\ControlFlowTok{function}\NormalTok{(x) \{}\FunctionTok{extent}\NormalTok{(x)[}\DecValTok{1}\NormalTok{]\})))}
\NormalTok{    xmax }\OtherTok{\textless{}{-}} \FunctionTok{max}\NormalTok{(}\FunctionTok{unlist}\NormalTok{(}\FunctionTok{lapply}\NormalTok{(raster.list, }\ControlFlowTok{function}\NormalTok{(x) \{}\FunctionTok{extent}\NormalTok{(x)[}\DecValTok{2}\NormalTok{]\})))}
\NormalTok{    ymin }\OtherTok{\textless{}{-}} \FunctionTok{max}\NormalTok{(}\FunctionTok{unlist}\NormalTok{(}\FunctionTok{lapply}\NormalTok{(raster.list, }\ControlFlowTok{function}\NormalTok{(x) \{}\FunctionTok{extent}\NormalTok{(x)[}\DecValTok{3}\NormalTok{]\})))}
\NormalTok{    ymax }\OtherTok{\textless{}{-}} \FunctionTok{max}\NormalTok{(}\FunctionTok{unlist}\NormalTok{(}\FunctionTok{lapply}\NormalTok{(raster.list, }\ControlFlowTok{function}\NormalTok{(x) \{}\FunctionTok{extent}\NormalTok{(x)[}\DecValTok{4}\NormalTok{]\})))}
\NormalTok{  \}}
  
  \CommentTok{\# Create new raster }
\NormalTok{  mock.surface }\OtherTok{\textless{}{-}} \FunctionTok{raster}\NormalTok{(}
    \AttributeTok{ncol=}\NormalTok{(xmax}\SpecialCharTok{{-}}\NormalTok{xmin)}\SpecialCharTok{/}\NormalTok{resolution}\SpecialCharTok{$}\NormalTok{x, }\CommentTok{\# raster automatically rounds, total cols}
    \AttributeTok{nrow=}\NormalTok{(ymax}\SpecialCharTok{{-}}\NormalTok{ymin)}\SpecialCharTok{/}\NormalTok{resolution}\SpecialCharTok{$}\NormalTok{y, }\CommentTok{\# raster automatically rounds, total rows}
    \AttributeTok{xmn=}\NormalTok{xmin, }\CommentTok{\# min x exent}
    \AttributeTok{xmx=}\NormalTok{xmax, }\CommentTok{\# max x exent}
    \AttributeTok{ymn=}\NormalTok{ymin, }\CommentTok{\# min y exent}
    \AttributeTok{ymx=}\NormalTok{ymax, }\CommentTok{\# max y exent}
    \AttributeTok{crs =} \FunctionTok{crs}\NormalTok{(raster.list[[}\DecValTok{1}\NormalTok{]])) }\CommentTok{\# take CRS from first raster, requires that rasters match in CRS}
  
  \FunctionTok{values}\NormalTok{(mock.surface) }\OtherTok{\textless{}{-}} \DecValTok{1}\SpecialCharTok{:}\NormalTok{(}\FunctionTok{ncol}\NormalTok{(mock.surface)}\SpecialCharTok{*}\FunctionTok{nrow}\NormalTok{(mock.surface)) }\CommentTok{\# not important, just for visualization}
  
  \FunctionTok{return}\NormalTok{(mock.surface)}
\NormalTok{\}}
\end{Highlighting}
\end{Shaded}

\hypertarget{check-model-input}{%
\subsection{Check model input}\label{check-model-input}}

We should check that a model is correctly specified, before throwing
errors down the line.

\begin{Shaded}
\begin{Highlighting}[]
\NormalTok{check\_ssf }\OtherTok{\textless{}{-}} \ControlFlowTok{function}\NormalTok{(ssf.obj) \{}
  \FunctionTok{inherits}\NormalTok{(ssf.obj, }\FunctionTok{c}\NormalTok{(}\StringTok{"fit\_clogit"}\NormalTok{)) }\CommentTok{\# not broad enough, basically just from smt at the moment}
\NormalTok{\}}
\end{Highlighting}
\end{Shaded}

\hypertarget{find-reasonable-step-distances-for-neighborhoods-down-the-line}{%
\subsection{Find reasonable step distances for neighborhoods down the
line}\label{find-reasonable-step-distances-for-neighborhoods-down-the-line}}

We should create a reasonable null step, and we can do so by using the
estimated gamma distribution from amt. However, we could just as easily
take quantiles from the distribution of step lengths we observe,
instead.

\begin{Shaded}
\begin{Highlighting}[]
\NormalTok{step\_distance }\OtherTok{\textless{}{-}} \ControlFlowTok{function}\NormalTok{(ssf.obj, quantile) \{}
  \ControlFlowTok{if}\NormalTok{(}\SpecialCharTok{!}\FunctionTok{check\_ssf}\NormalTok{(ssf.obj)) }\FunctionTok{stop}\NormalTok{(}\StringTok{"Check that SSF model is valid"}\NormalTok{)}
  
\NormalTok{  step }\OtherTok{\textless{}{-}} \FunctionTok{qgamma}\NormalTok{(quantile, }\CommentTok{\# user specified quantile}
                 \AttributeTok{shape =}\NormalTok{ ssf.obj}\SpecialCharTok{$}\NormalTok{sl\_}\SpecialCharTok{$}\NormalTok{params}\SpecialCharTok{$}\NormalTok{shape, }\CommentTok{\# estimated from amt}
                 \AttributeTok{scale =}\NormalTok{ ssf.obj}\SpecialCharTok{$}\NormalTok{sl\_}\SpecialCharTok{$}\NormalTok{params}\SpecialCharTok{$}\NormalTok{scale) }\CommentTok{\# estimated from amt}
  
  \FunctionTok{return}\NormalTok{(step)}
\NormalTok{\}}
\end{Highlighting}
\end{Shaded}

\hypertarget{get-data-from-prediction-surface}{%
\subsection{Get data from prediction
surface}\label{get-data-from-prediction-surface}}

Now that we have a valid SSF object and a prediction surface, we need to
find our prediction cells of interest. Lets get our raster data.

\begin{Shaded}
\begin{Highlighting}[]
\NormalTok{get\_cells }\OtherTok{\textless{}{-}} \ControlFlowTok{function}\NormalTok{(ssf.obj, mock.surface, raster, }\AttributeTok{accessory.x.preds =} \ConstantTok{NULL}\NormalTok{)\{}
  \ControlFlowTok{if}\NormalTok{(}\SpecialCharTok{!}\FunctionTok{check\_ssf}\NormalTok{(ssf.obj)) }\FunctionTok{stop}\NormalTok{(}\StringTok{"Check that SSF model is valid"}\NormalTok{)}
  
\NormalTok{  pred.xy }\OtherTok{\textless{}{-}}\NormalTok{ raster}\SpecialCharTok{::}\FunctionTok{coordinates}\NormalTok{(mock.surface) }\CommentTok{\# get coordinates from grid we created}
  
\NormalTok{  predict.data }\OtherTok{\textless{}{-}} \FunctionTok{data.frame}\NormalTok{(}\FunctionTok{cbind}\NormalTok{(pred.xy, raster}\SpecialCharTok{::}\FunctionTok{extract}\NormalTok{(raster, pred.xy, }\AttributeTok{df=}\ConstantTok{TRUE}\NormalTok{))) }\CommentTok{\# makes raster values a data frame}
  
\NormalTok{  predict.data}\SpecialCharTok{$}\NormalTok{step\_id\_unique }\OtherTok{=}\NormalTok{ ssf.obj}\SpecialCharTok{$}\NormalTok{model}\SpecialCharTok{$}\NormalTok{xlevels}\SpecialCharTok{$}\StringTok{\textasciigrave{}}\AttributeTok{strata(step\_id\_)}\StringTok{\textasciigrave{}}\NormalTok{[}\DecValTok{1}\NormalTok{] }\CommentTok{\# fix the strata to something reasonable}
  
  \ControlFlowTok{if}\NormalTok{(}\SpecialCharTok{!}\FunctionTok{is.null}\NormalTok{(accessory.x.preds)) \{}
\NormalTok{    predict.data }\OtherTok{\textless{}{-}} \FunctionTok{cbind}\NormalTok{(predict.data, accessory.x.preds) }\CommentTok{\# this adds extraneous x values that are not in matrix}
\NormalTok{  \}}
  
\NormalTok{  cells }\OtherTok{\textless{}{-}} \FunctionTok{nrow}\NormalTok{(pred.xy) }\CommentTok{\# number of cells}
  
\NormalTok{  predict.data}\SpecialCharTok{$}\NormalTok{cellnr }\OtherTok{\textless{}{-}} \DecValTok{1}\SpecialCharTok{:}\NormalTok{cells }\CommentTok{\# assign cell numbers, redundant of ID}
  
  \FunctionTok{return}\NormalTok{(predict.data)}
\NormalTok{\}}
\end{Highlighting}
\end{Shaded}

\hypertarget{check-possible-prediction-surfaces}{%
\subsection{Check possible prediction
surfaces}\label{check-possible-prediction-surfaces}}

Not all combinations of parameter space are valid, and we often are
missing raster data at edges

\begin{Shaded}
\begin{Highlighting}[]
\NormalTok{get\_cell\_data }\OtherTok{\textless{}{-}} \ControlFlowTok{function}\NormalTok{(ssf.obj, pred.data)\{}
  \ControlFlowTok{if}\NormalTok{(}\SpecialCharTok{!}\FunctionTok{check\_ssf}\NormalTok{(ssf.obj)) }\FunctionTok{stop}\NormalTok{(}\StringTok{"Check that SSF model is valid"}\NormalTok{)}
  
\NormalTok{  cells }\OtherTok{\textless{}{-}} \FunctionTok{nrow}\NormalTok{(pred.data) }\CommentTok{\# number of cells}
  
  \CommentTok{\# relying on amt step{-}selection models, we can predict log{-}RSS}
  \CommentTok{\# there are better ways to predict, but this is simple}
\NormalTok{  log.rss }\OtherTok{\textless{}{-}}\NormalTok{ amt}\SpecialCharTok{::}\FunctionTok{log\_rss}\NormalTok{(ssf.obj, }\CommentTok{\# the model}
\NormalTok{                          pred.data, }\CommentTok{\# the raster data (including missing values)}
\NormalTok{                          pred.data }\SpecialCharTok{\%\textgreater{}\%} 
                            \FunctionTok{drop\_na}\NormalTok{() }\SpecialCharTok{\%\textgreater{}\%} 
                            \FunctionTok{sample\_n}\NormalTok{(}\DecValTok{1}\NormalTok{),  }\CommentTok{\# a row of the raster data (excluding missing values)}
                          \AttributeTok{ci =} \ConstantTok{NA}\NormalTok{) }
  
\NormalTok{  full.raster.data }\OtherTok{\textless{}{-}} \FunctionTok{data.frame}\NormalTok{(pred.data, }\AttributeTok{lRSS =}\NormalTok{ log.rss}\SpecialCharTok{$}\NormalTok{df}\SpecialCharTok{$}\NormalTok{log\_rss) }\CommentTok{\# bind predictions to the original data}
  
  \FunctionTok{return}\NormalTok{(full.raster.data)}
  
\NormalTok{\}}
\end{Highlighting}
\end{Shaded}

\hypertarget{we-need-a-quick-way-to-find-values-for-comparison}{%
\subsection{We need a quick way to find values for
comparison}\label{we-need-a-quick-way-to-find-values-for-comparison}}

We need to find neighbors of cells in our matrix. This is easy, but
could require n\^{}3 comparisons. So many comparisons are
computationally inefficient. One thing we can rely on is that all cells
in a raster (or matrix) can be indexed. Additionally, distances between
cells in a raster are repetitive. In general, there are only
\textasciitilde0.2\% unique pairwise distances between cell indices. We
can find these unique distances because we know the difference in index
of the comparisons, the column identify of the minimum index, and the
observed distance. This allows us to have a table that we can call
repetitively instead of recalculating millions of distances.

\begin{Shaded}
\begin{Highlighting}[]
\NormalTok{neighbor\_lookup }\OtherTok{\textless{}{-}} \ControlFlowTok{function}\NormalTok{(mock.surface, cell.data, }\AttributeTok{cell.data.list =} \ConstantTok{NULL}\NormalTok{)\{}
\NormalTok{  cols }\OtherTok{\textless{}{-}}\NormalTok{ mock.surface}\SpecialCharTok{@}\NormalTok{ncols }\CommentTok{\# columns in prediction}
\NormalTok{  rows }\OtherTok{\textless{}{-}}\NormalTok{ mock.surface}\SpecialCharTok{@}\NormalTok{nrows }\CommentTok{\# rows in prediction}
\NormalTok{  cells }\OtherTok{\textless{}{-}}\NormalTok{ cols}\SpecialCharTok{*}\NormalTok{rows }\CommentTok{\# number of cells}
\NormalTok{  index }\OtherTok{\textless{}{-}} \DecValTok{1}\SpecialCharTok{:}\NormalTok{cells }\CommentTok{\# all index values in our prediction raster}
  
  \CommentTok{\# create a matrix for each column in the first row and its comparisons to distance to all other cells }
  
  \ControlFlowTok{if}\NormalTok{(}\FunctionTok{is.null}\NormalTok{(cell.data.list))\{}
    \FunctionTok{print}\NormalTok{(}\StringTok{"Splitting cell.data into list"}\NormalTok{)}
\NormalTok{    cell.data.list }\OtherTok{\textless{}{-}} \FunctionTok{split}\NormalTok{(cell.data, cell.data}\SpecialCharTok{$}\NormalTok{cellnr) }\CommentTok{\# split the prediction data into row{-}wise lists to use lapply}
\NormalTok{  \}}
  
  \ControlFlowTok{if}\NormalTok{(}\SpecialCharTok{!}\FunctionTok{is.null}\NormalTok{(cell.data.list))\{}
    \FunctionTok{print}\NormalTok{(}\StringTok{"Using inputted list of cell data"}\NormalTok{)}
\NormalTok{    cell.data.list }\OtherTok{\textless{}{-}}\NormalTok{ cell.data.list }\CommentTok{\# split the prediction data into row{-}wise lists to use lapply}
\NormalTok{  \}}
    
  \CommentTok{\# this is a progress bar we can use in a for loop}
\NormalTok{  pb }\OtherTok{\textless{}{-}}\NormalTok{ progress\_bar}\SpecialCharTok{$}\FunctionTok{new}\NormalTok{(}\AttributeTok{format =} \StringTok{"(:spin) [:bar] :percent [Elapsed time: :elapsedfull || Estimated time remaining: :eta]"}\NormalTok{, }\AttributeTok{total =}\NormalTok{ cols, }\AttributeTok{complete =} \StringTok{"="}\NormalTok{, }\AttributeTok{incomplete =} \StringTok{"{-}"}\NormalTok{, }\AttributeTok{current =} \StringTok{"\textgreater{}"}\NormalTok{, }\AttributeTok{clear =} \ConstantTok{FALSE}\NormalTok{, }\AttributeTok{width =} \DecValTok{100}\NormalTok{)}
  
  \ControlFlowTok{for}\NormalTok{(i }\ControlFlowTok{in} \DecValTok{1}\SpecialCharTok{:}\NormalTok{cols) \{ }\CommentTok{\# step through columns}
\NormalTok{    dist }\OtherTok{\textless{}{-}} \FunctionTok{pointDistance}\NormalTok{(cell.data[i,}\FunctionTok{c}\NormalTok{(}\StringTok{"x"}\NormalTok{,}\StringTok{"y"}\NormalTok{)], }\CommentTok{\# choose the first row cell by column (raster indices are row{-}wise)}
\NormalTok{                            cell.data[i}\SpecialCharTok{:}\NormalTok{cells,}\FunctionTok{c}\NormalTok{(}\StringTok{"x"}\NormalTok{,}\StringTok{"y"}\NormalTok{)], }\CommentTok{\# choose all other cells}
                            \AttributeTok{lonlat =}\NormalTok{ F) }\CommentTok{\# we are using UTM }
    \ControlFlowTok{if}\NormalTok{(i }\SpecialCharTok{==} \DecValTok{1}\NormalTok{) mat.dist }\OtherTok{\textless{}{-}} \FunctionTok{matrix}\NormalTok{(dist, }\AttributeTok{ncol =} \DecValTok{1}\NormalTok{)}
    \ControlFlowTok{if}\NormalTok{(i }\SpecialCharTok{\textgreater{}} \DecValTok{1}\NormalTok{) mat.dist }\OtherTok{\textless{}{-}} \FunctionTok{cbind}\NormalTok{(mat.dist, }\FunctionTok{c}\NormalTok{(dist, }\FunctionTok{rep}\NormalTok{(}\ConstantTok{NA}\NormalTok{, i}\DecValTok{{-}1}\NormalTok{))) }\CommentTok{\# for each additional column, there are i{-}1 comparisons that are repeated (unnecessary)}
\NormalTok{    pb}\SpecialCharTok{$}\FunctionTok{tick}\NormalTok{() }\CommentTok{\# for progress bar}
\NormalTok{  \}}
  
  \FunctionTok{return}\NormalTok{(mat.dist)}
\NormalTok{\}}
\end{Highlighting}
\end{Shaded}

\hypertarget{find-neighbors-within-distance}{%
\subsection{Find neighbors within
distance}\label{find-neighbors-within-distance}}

Now that we have our call-up table, we can use it to generate neighbors.
This is arguably one of the most taxing (computationally) steps of the
whole process. Luckily im a damn genius and found ways to make this even
faster than the call-up table.

\begin{Shaded}
\begin{Highlighting}[]
\NormalTok{neighbor\_finder }\OtherTok{\textless{}{-}} \ControlFlowTok{function}\NormalTok{(}\AttributeTok{ssf.obj =}\NormalTok{ m2, cell.data, neighbors.found, }\AttributeTok{quantile =} \FloatTok{0.99}\NormalTok{, }\AttributeTok{cell.data.list =} \ConstantTok{NULL}\NormalTok{)\{}
  
\NormalTok{  neighborhood.distance }\OtherTok{\textless{}{-}} \FunctionTok{step\_distance}\NormalTok{(ssf.obj, quantile) }\CommentTok{\# take the X\% step distance as your neighborhood }
  
\NormalTok{  cols }\OtherTok{\textless{}{-}} \FunctionTok{ncol}\NormalTok{(neighbors.found) }\CommentTok{\# columns of our call{-}up table}
\NormalTok{  differences }\OtherTok{\textless{}{-}} \FunctionTok{nrow}\NormalTok{(neighbors.found) }\CommentTok{\# number of differences in index values }
  
  \FunctionTok{print}\NormalTok{(}\StringTok{"Creating neighbor comparisons"}\NormalTok{)}
\NormalTok{  vector }\OtherTok{\textless{}{-}} \FunctionTok{c}\NormalTok{(neighbors.found) }\CommentTok{\# convert the neighbors to a vector we can index later, this is the purpose of all those NA\textquotesingle{}s earlier based on i{-}1 unique distances}
  
  \FunctionTok{print}\NormalTok{(}\StringTok{"Finding valid comparisons"}\NormalTok{)}
\NormalTok{  valid }\OtherTok{\textless{}{-}}\NormalTok{ vector }\SpecialCharTok{\textless{}}\NormalTok{ neighborhood.distance }\CommentTok{\# T/F whether those neighborhood distances are less than our threshold}
  
  \ControlFlowTok{if}\NormalTok{(}\FunctionTok{is.null}\NormalTok{(cell.data.list))\{}
    \FunctionTok{print}\NormalTok{(}\StringTok{"Splitting cell.data into list"}\NormalTok{)}
\NormalTok{    cell.data.list }\OtherTok{\textless{}{-}} \FunctionTok{split}\NormalTok{(cell.data, cell.data}\SpecialCharTok{$}\NormalTok{cellnr) }\CommentTok{\# split the prediction data into row{-}wise lists to use lapply}
\NormalTok{  \}}
  
  \ControlFlowTok{if}\NormalTok{(}\SpecialCharTok{!}\FunctionTok{is.null}\NormalTok{(cell.data.list))\{}
    \FunctionTok{print}\NormalTok{(}\StringTok{"Using inputted list of cell data"}\NormalTok{)}
\NormalTok{    cell.data.list }\OtherTok{\textless{}{-}}\NormalTok{ cell.data.list }\CommentTok{\# split the prediction data into row{-}wise lists to use lapply}
\NormalTok{  \}}
  
  \FunctionTok{print}\NormalTok{(}\StringTok{"Running comparisons"}\NormalTok{)}
\NormalTok{  neighbor.mat }\OtherTok{\textless{}{-}} \FunctionTok{pblapply}\NormalTok{(cell.data.list, }\ControlFlowTok{function}\NormalTok{(x)\{ }\CommentTok{\# step through each row (see list split above)}
\NormalTok{    focal }\OtherTok{\textless{}{-}} \FunctionTok{as.numeric}\NormalTok{(x}\SpecialCharTok{$}\NormalTok{cellnr) }\CommentTok{\# value of row cell number }
\NormalTok{    delta }\OtherTok{\textless{}{-}} \FunctionTok{abs}\NormalTok{(focal }\SpecialCharTok{{-}}\NormalTok{ cell.data}\SpecialCharTok{$}\NormalTok{cellnr) }\CommentTok{\# difference in row cell number vs all others}
\NormalTok{    index }\OtherTok{\textless{}{-}} \FunctionTok{ifelse}\NormalTok{(focal }\SpecialCharTok{\textless{}}\NormalTok{ cell.data}\SpecialCharTok{$}\NormalTok{cellnr, focal, cell.data}\SpecialCharTok{$}\NormalTok{cellnr) }\CommentTok{\# report the minimum cell index (based on our call{-}up structure)}
    
\NormalTok{    index.col }\OtherTok{\textless{}{-}}\NormalTok{ index}\SpecialCharTok{\%\%}\NormalTok{cols }\CommentTok{\# use the remainder function to get the column number (see below, we have to force zeros to the column number because the remainder of the final column is zero)}
    
\NormalTok{    df }\OtherTok{\textless{}{-}} \FunctionTok{data.frame}\NormalTok{(}\AttributeTok{difference =}\NormalTok{ delta }\SpecialCharTok{+} \DecValTok{1}\NormalTok{, }\CommentTok{\# we have to add one because differences of 0 are stored in row 1, differences of 1 in row 2, etc. }
                     \AttributeTok{col =} \FunctionTok{ifelse}\NormalTok{(index.col }\SpecialCharTok{==} \DecValTok{0}\NormalTok{, cols, index.col), }\CommentTok{\# forcing remainders of zero the number of columns}
                     \AttributeTok{cell.nr =}\NormalTok{ cell.data}\SpecialCharTok{$}\NormalTok{cellnr) }\CommentTok{\# just tracking the cell number we are comparing against for use later}
    
    \CommentTok{\# filter the data frame based on whether our call{-}up values are less than the neighborhood}
\NormalTok{    df }\OtherTok{\textless{}{-}}\NormalTok{ df }\SpecialCharTok{\%\textgreater{}\%} 
      \FunctionTok{filter}\NormalTok{(valid[difference}\SpecialCharTok{+}\NormalTok{((col}\DecValTok{{-}1}\NormalTok{)}\SpecialCharTok{*}\NormalTok{differences)]) }
    
    \CommentTok{\# filter the data set to unique rows and columns for call{-}up (to accommodate memory issues)}
\NormalTok{    df.distinct }\OtherTok{\textless{}{-}}\NormalTok{ df }\SpecialCharTok{\%\textgreater{}\%} \FunctionTok{distinct}\NormalTok{(difference, col) }
    
    \CommentTok{\# find the unique distances (trims time down)}
\NormalTok{    df.distinct}\SpecialCharTok{$}\NormalTok{distances }\OtherTok{\textless{}{-}}\NormalTok{ vector[df.distinct}\SpecialCharTok{$}\NormalTok{difference }\SpecialCharTok{+}\NormalTok{ ((df.distinct}\SpecialCharTok{$}\NormalTok{col }\SpecialCharTok{{-}} \DecValTok{1}\NormalTok{)}\SpecialCharTok{*}\NormalTok{differences)]}
    
    \CommentTok{\# throw the unique distances back to the full data set }
\NormalTok{    df }\OtherTok{\textless{}{-}} \FunctionTok{merge}\NormalTok{(df, df.distinct, }\AttributeTok{by =} \FunctionTok{c}\NormalTok{(}\StringTok{"difference"}\NormalTok{,}\StringTok{"col"}\NormalTok{))}
    
    \CommentTok{\# package into a nice data frame for export}
    \FunctionTok{data.frame}\NormalTok{(}\AttributeTok{row =}\NormalTok{ focal, }\AttributeTok{column =}\NormalTok{ df}\SpecialCharTok{$}\NormalTok{cell.nr, }\AttributeTok{distance =}\NormalTok{ df}\SpecialCharTok{$}\NormalTok{distances)}
\NormalTok{  \})}
  
  \CommentTok{\# bind the output list}
\NormalTok{  neighbors }\OtherTok{\textless{}{-}} \FunctionTok{rbindlist}\NormalTok{(neighbor.mat)}
  
  \CommentTok{\# create a sparse matrix based on the focal id, alternate id, and distance}
\NormalTok{  sparse.neighbors }\OtherTok{\textless{}{-}} \FunctionTok{sparseMatrix}\NormalTok{(}\AttributeTok{i =}\NormalTok{ neighbors}\SpecialCharTok{$}\NormalTok{row, }\AttributeTok{j =}\NormalTok{ neighbors}\SpecialCharTok{$}\NormalTok{column, }\AttributeTok{x =}\NormalTok{ neighbors}\SpecialCharTok{$}\NormalTok{distance)}
  
  \CommentTok{\# return both the matrix and the unbound list of neighbor cells}
  \FunctionTok{return}\NormalTok{(}\FunctionTok{list}\NormalTok{(}\AttributeTok{matrix =}\NormalTok{ sparse.neighbors, }\AttributeTok{by.cell =}\NormalTok{ neighbor.mat))}
\NormalTok{\}}
\end{Highlighting}
\end{Shaded}

\hypertarget{we-need-to-split-data-into-focal-and-neighbor-groups}{%
\subsection{We need to split data into focal and neighbor
groups}\label{we-need-to-split-data-into-focal-and-neighbor-groups}}

Now that we have data divided so that we know the neighbors of a focal
cell, we can split the data into a format that is conducive to SSF
predictions. We split these into two data frames, \texttt{.given} for
the focal cell and \texttt{.for} for the neighboring cells.

\begin{Shaded}
\begin{Highlighting}[]
\NormalTok{compile\_ssf\_comparisons }\OtherTok{\textless{}{-}} \ControlFlowTok{function}\NormalTok{(sparse.neighbors, cell.data) \{}
  
  \CommentTok{\# this is why the export of the neighbors as individual lists was important}
\NormalTok{  ssf.comparisons }\OtherTok{\textless{}{-}} \FunctionTok{lapply}\NormalTok{(sparse.neighbors}\SpecialCharTok{$}\NormalTok{by.cell, }\ControlFlowTok{function}\NormalTok{(x)\{ }
    
\NormalTok{    baseline }\OtherTok{\textless{}{-}}\NormalTok{ cell.data[x}\SpecialCharTok{$}\NormalTok{column[}\FunctionTok{which}\NormalTok{(x}\SpecialCharTok{$}\NormalTok{distance }\SpecialCharTok{==} \DecValTok{0}\NormalTok{)],] }\CommentTok{\# baseline will have a distance of zero (focal)}
    
\NormalTok{    baseline}\SpecialCharTok{$}\NormalTok{step }\OtherTok{\textless{}{-}} \DecValTok{0} \CommentTok{\# create a variable "step" that records this zero distance}
    
\NormalTok{    alternate }\OtherTok{\textless{}{-}}\NormalTok{ cell.data[x}\SpecialCharTok{$}\NormalTok{column,] }\CommentTok{\# grab all the other cell.data for neighboring cells (including focal cell)}
    
\NormalTok{    alternate}\SpecialCharTok{$}\NormalTok{step }\OtherTok{\textless{}{-}}\NormalTok{ x}\SpecialCharTok{$}\NormalTok{distance }\CommentTok{\# force distance to this new variable step}
    
    \FunctionTok{list}\NormalTok{(}\AttributeTok{.given =}\NormalTok{ baseline, }\AttributeTok{.for =}\NormalTok{ alternate) }\CommentTok{\# return a list of focal and neigboring cell data}
    
\NormalTok{  \})}
  
  \FunctionTok{return}\NormalTok{(ssf.comparisons)}
\NormalTok{\}}
\end{Highlighting}
\end{Shaded}

\hypertarget{now-we-need-to-predict-our-surface}{%
\subsection{Now we need to predict our
surface}\label{now-we-need-to-predict-our-surface}}

Using our fit movement model and the metadata for our prediction
surface, we can estimate the relative risk of selecting the focal cell
and all other cells in the neighborhood. We can show that all
probabilities of ``choosing'' a cell in the prediction surface must sum
to one, thus the probability of selecting the focal cell is the inverse
of he sum of all relative probabilites. From log-RSS, we just
exponentiate, and take the inverse sum. To find the probability of
choosing all cells, we just multiply the probability of selecting the
focal cell against all relative risks! Easy! This tells us the
probability of selecting each of those cells given the comparison to the
focal cell, so not the out-right probability of selection, but it gets
us closer.

\begin{Shaded}
\begin{Highlighting}[]
\NormalTok{predict\_ssf\_comparisons }\OtherTok{\textless{}{-}} \ControlFlowTok{function}\NormalTok{(}\AttributeTok{ssf.obj =}\NormalTok{ m2, ssf.comparisons) \{}
  
  \FunctionTok{print}\NormalTok{(}\StringTok{"Estimating probability surface"}\NormalTok{)}
  
\NormalTok{  prediction.list }\OtherTok{\textless{}{-}} \FunctionTok{pbmclapply}\NormalTok{(ssf.comparisons, }\ControlFlowTok{function}\NormalTok{(x)\{ }\CommentTok{\# step through the list of SSF objects}
\NormalTok{    log.rss }\OtherTok{\textless{}{-}} \FunctionTok{log\_rss}\NormalTok{(ssf.obj, x}\SpecialCharTok{$}\NormalTok{.for, x}\SpecialCharTok{$}\NormalTok{.given, }\AttributeTok{ci =} \ConstantTok{NA}\NormalTok{) }\CommentTok{\# get the log{-}RSS for each comparison}
\NormalTok{    x}\SpecialCharTok{$}\NormalTok{.for}\SpecialCharTok{$}\NormalTok{Prob }\OtherTok{\textless{}{-}} \FunctionTok{exp}\NormalTok{(log.rss}\SpecialCharTok{$}\NormalTok{df}\SpecialCharTok{$}\NormalTok{log\_rss)}\SpecialCharTok{*}\NormalTok{(}\DecValTok{1}\SpecialCharTok{/}\FunctionTok{sum}\NormalTok{(}\FunctionTok{exp}\NormalTok{(log.rss}\SpecialCharTok{$}\NormalTok{df}\SpecialCharTok{$}\NormalTok{log\_rss))) }\CommentTok{\# exponentiate and multiply agains relative risk}
\NormalTok{    x}\SpecialCharTok{$}\NormalTok{.for }\CommentTok{\# return the data frame with probabilities}
\NormalTok{  \})}
  
  \FunctionTok{print}\NormalTok{(}\StringTok{"Compiling probability surface"}\NormalTok{)}
  
  \ControlFlowTok{for}\NormalTok{(i }\ControlFlowTok{in} \DecValTok{1}\SpecialCharTok{:}\FunctionTok{length}\NormalTok{(prediction.list))\{}
\NormalTok{    prediction.list[[i]]}\SpecialCharTok{$}\NormalTok{focal.cell }\OtherTok{\textless{}{-}}\NormalTok{ i }\CommentTok{\# specify the focal cell for each comparison}
\NormalTok{  \} }
  
  \FunctionTok{print}\NormalTok{(}\StringTok{"Making sparse matrix for transitions"}\NormalTok{)}
\NormalTok{  bound }\OtherTok{\textless{}{-}} \FunctionTok{rbindlist}\NormalTok{(prediction.list) }\CommentTok{\# bind all data frames }
  
  \CommentTok{\# use indexing to make a massive sparse matrix quickly }
\NormalTok{  Sparse.Matrix }\OtherTok{\textless{}{-}} \FunctionTok{sparseMatrix}\NormalTok{(bound}\SpecialCharTok{$}\NormalTok{focal.cell, bound}\SpecialCharTok{$}\NormalTok{cellnr, }\AttributeTok{x =}\NormalTok{ bound}\SpecialCharTok{$}\NormalTok{Prob) }
  
  \CommentTok{\# return the prediction list and sparse matrix}
  \FunctionTok{return}\NormalTok{(}\FunctionTok{list}\NormalTok{(}\AttributeTok{prob.surface =}\NormalTok{ prediction.list, }\AttributeTok{sparse.matrix =}\NormalTok{ Sparse.Matrix))  }
\NormalTok{\}}
\end{Highlighting}
\end{Shaded}

\hypertarget{package-applications}{%
\section{Package applications(?)}\label{package-applications}}

\hypertarget{deer}{%
\subsection{Deer}\label{deer}}

This is a demonstration data set from the amt package. Single deer from
Northern Europe and a binary forest layer as a covariate.

\hypertarget{data}{%
\subsubsection{Data}\label{data}}

We read in data, create 15 random steps for each real step, and then
store forest as a factor, turning angle as cosine(ta) and step length as
log plus 1 (this saves us later when we compare to a focal cell with
``0'' step distance)

\begin{Shaded}
\begin{Highlighting}[]
\FunctionTok{data}\NormalTok{(}\StringTok{"deer"}\NormalTok{)}
\FunctionTok{data}\NormalTok{(}\StringTok{"sh\_forest"}\NormalTok{)}
\NormalTok{ssf1 }\OtherTok{\textless{}{-}}\NormalTok{ deer }\SpecialCharTok{\%\textgreater{}\%} 
  \FunctionTok{steps\_by\_burst}\NormalTok{() }\SpecialCharTok{\%\textgreater{}\%} 
  \FunctionTok{random\_steps}\NormalTok{(}\AttributeTok{n\_control =} \DecValTok{15}\NormalTok{) }\SpecialCharTok{\%\textgreater{}\%} 
  \FunctionTok{extract\_covariates}\NormalTok{(sh\_forest) }\SpecialCharTok{\%\textgreater{}\%}
  \FunctionTok{mutate}\NormalTok{(}\AttributeTok{sh.forest =} \FunctionTok{factor}\NormalTok{(sh.forest),}
         \AttributeTok{cos\_ta =} \FunctionTok{cos}\NormalTok{(ta\_),}
         \AttributeTok{log\_sl =} \FunctionTok{log}\NormalTok{(sl\_}\SpecialCharTok{+}\DecValTok{1}\NormalTok{))}

\FunctionTok{par}\NormalTok{(}\AttributeTok{mfrow =} \FunctionTok{c}\NormalTok{(}\DecValTok{1}\NormalTok{,}\DecValTok{2}\NormalTok{))}
\FunctionTok{plot}\NormalTok{(}\DecValTok{1}\SpecialCharTok{{-}}\NormalTok{(sh\_forest}\DecValTok{{-}1}\NormalTok{))}
\FunctionTok{plot}\NormalTok{(ssf1)}
\end{Highlighting}
\end{Shaded}

\includegraphics{Untitled_files/figure-latex/unnamed-chunk-11-1.pdf}

\hypertarget{model}{%
\subsubsection{Model}\label{model}}

We can fit a basic movement model; it appears that a forest:step
interaction is favored.

\begin{Shaded}
\begin{Highlighting}[]
\NormalTok{m2 }\OtherTok{\textless{}{-}}\NormalTok{ ssf1 }\SpecialCharTok{\%\textgreater{}\%}
  \FunctionTok{filter}\NormalTok{(}\SpecialCharTok{!}\FunctionTok{is.na}\NormalTok{(cos\_ta)) }\SpecialCharTok{\%\textgreater{}\%} 
  \FunctionTok{fit\_clogit}\NormalTok{(case\_ }\SpecialCharTok{\textasciitilde{}}\NormalTok{ sh.forest }\SpecialCharTok{*} \FunctionTok{poly}\NormalTok{(log\_sl,}\DecValTok{2}\NormalTok{) }\SpecialCharTok{+} \FunctionTok{strata}\NormalTok{(step\_id\_), }\AttributeTok{model =}\NormalTok{ T)}
\FunctionTok{summary}\NormalTok{(m2)}
\end{Highlighting}
\end{Shaded}

\begin{verbatim}
## Call:
## coxph(formula = Surv(rep(1, 12624L), case_) ~ sh.forest * poly(log_sl, 
##     2) + strata(step_id_), data = data, model = ..1, method = "exact")
## 
##   n= 12624, number of events= 759 
## 
##                                   coef  exp(coef)   se(coef)      z Pr(>|z|)
## sh.forest2                  -4.371e-01  6.459e-01  1.103e-01 -3.962 7.44e-05
## poly(log_sl, 2)1             2.720e+01  6.475e+11  8.051e+00  3.378  0.00073
## poly(log_sl, 2)2             7.680e+00  2.165e+03  8.010e+00  0.959  0.33761
## sh.forest2:poly(log_sl, 2)1 -4.020e+01  3.468e-18  9.872e+00 -4.072 4.65e-05
## sh.forest2:poly(log_sl, 2)2 -3.018e+01  7.806e-14  9.974e+00 -3.026  0.00248
##                                
## sh.forest2                  ***
## poly(log_sl, 2)1            ***
## poly(log_sl, 2)2               
## sh.forest2:poly(log_sl, 2)1 ***
## sh.forest2:poly(log_sl, 2)2 ** 
## ---
## Signif. codes:  0 '***' 0.001 '**' 0.01 '*' 0.05 '.' 0.1 ' ' 1
## 
##                             exp(coef) exp(-coef) lower .95 upper .95
## sh.forest2                  6.459e-01  1.548e+00 5.203e-01 8.018e-01
## poly(log_sl, 2)1            6.475e+11  1.544e-12 9.084e+04 4.616e+18
## poly(log_sl, 2)2            2.165e+03  4.618e-04 3.295e-04 1.423e+10
## sh.forest2:poly(log_sl, 2)1 3.468e-18  2.883e+17 1.371e-26 8.770e-10
## sh.forest2:poly(log_sl, 2)2 7.806e-14  1.281e+13 2.528e-22 2.410e-05
## 
## Concordance= 0.59  (se = 0.012 )
## Likelihood ratio test= 54.31  on 5 df,   p=2e-10
## Wald test            = 53.28  on 5 df,   p=3e-10
## Score (logrank) test = 54.73  on 5 df,   p=1e-10
\end{verbatim}

\hypertarget{surface}{%
\subsubsection{Surface}\label{surface}}

We can make our surface for predictions - rememeber, we set initial
values to their cell index number.

\begin{Shaded}
\begin{Highlighting}[]
\NormalTok{mock.surface }\OtherTok{\textless{}{-}} \FunctionTok{create\_mock\_surface}\NormalTok{(sh\_forest, F, }\FunctionTok{list}\NormalTok{(}\AttributeTok{x =} \DecValTok{250}\NormalTok{, }\AttributeTok{y =} \DecValTok{250}\NormalTok{))}
\FunctionTok{plot}\NormalTok{(mock.surface)}
\FunctionTok{lines}\NormalTok{(deer)}
\end{Highlighting}
\end{Shaded}

\includegraphics{Untitled_files/figure-latex/unnamed-chunk-13-1.pdf}

\hypertarget{getting-cell-data}{%
\subsubsection{Getting cell data}\label{getting-cell-data}}

We can get our prediction data. We can set forest to a factor again to
get the model predictions to work.

\begin{Shaded}
\begin{Highlighting}[]
\NormalTok{pred.data }\OtherTok{\textless{}{-}} \FunctionTok{get\_cells}\NormalTok{(m2, }
\NormalTok{                       mock.surface,}
\NormalTok{                       sh\_forest, }
                       \AttributeTok{accessory.x.preds =} \FunctionTok{list}\NormalTok{(}\AttributeTok{log\_sl =} \FunctionTok{log}\NormalTok{(}\FunctionTok{step\_distance}\NormalTok{(m2, }\FloatTok{0.5}\NormalTok{)), }
                                                \AttributeTok{cos\_ta =} \DecValTok{1}\NormalTok{))}

\NormalTok{pred.data}\SpecialCharTok{$}\NormalTok{sh.forest }\OtherTok{\textless{}{-}} \FunctionTok{factor}\NormalTok{(pred.data}\SpecialCharTok{$}\NormalTok{sh.forest)}
\FunctionTok{head}\NormalTok{(pred.data)}
\end{Highlighting}
\end{Shaded}

\begin{verbatim}
##         x       y ID sh.forest step_id_unique   log_sl cos_ta cellnr
## 1 4304850 3455600  1         1     step_id_=1 5.413297      1      1
## 2 4305100 3455600  2         2     step_id_=1 5.413297      1      2
## 3 4305351 3455600  3         2     step_id_=1 5.413297      1      3
## 4 4305601 3455600  4         2     step_id_=1 5.413297      1      4
## 5 4305852 3455600  5         2     step_id_=1 5.413297      1      5
## 6 4306102 3455600  6         2     step_id_=1 5.413297      1      6
\end{verbatim}

\hypertarget{testing-surface}{%
\subsubsection{Testing surface}\label{testing-surface}}

We can fit our original SSF sensu other publications, where a baseline
is chosen and run with. It looks like a RSF, but is it?

\begin{Shaded}
\begin{Highlighting}[]
\NormalTok{cell.data }\OtherTok{\textless{}{-}} \FunctionTok{get\_cell\_data}\NormalTok{(m2, pred.data)}
\FunctionTok{plot}\NormalTok{(}\FunctionTok{setValues}\NormalTok{(mock.surface, cell.data}\SpecialCharTok{$}\NormalTok{lRSS))}
\FunctionTok{points}\NormalTok{(deer, }\AttributeTok{pch =} \StringTok{"."}\NormalTok{, }\AttributeTok{col =} \FunctionTok{alpha}\NormalTok{(}\StringTok{"black"}\NormalTok{, }\FloatTok{0.5}\NormalTok{))}
\end{Highlighting}
\end{Shaded}

\includegraphics{Untitled_files/figure-latex/unnamed-chunk-15-1.pdf}

\hypertarget{finding-neighbors}{%
\subsubsection{Finding neighbors}\label{finding-neighbors}}

Lets make our neighbor look-up matrix. We can plot it here as a raster,
but its shows the general idea.

\begin{Shaded}
\begin{Highlighting}[]
\NormalTok{neighbors.found }\OtherTok{\textless{}{-}} \FunctionTok{neighbor\_lookup}\NormalTok{(mock.surface, cell.data)}
\end{Highlighting}
\end{Shaded}

\begin{verbatim}
## [1] "Splitting cell.data into list"
## [1] "Using inputted list of cell data"
\end{verbatim}

\begin{Shaded}
\begin{Highlighting}[]
\FunctionTok{plot}\NormalTok{(}\FunctionTok{raster}\NormalTok{(neighbors.found))}
\end{Highlighting}
\end{Shaded}

\includegraphics{Untitled_files/figure-latex/unnamed-chunk-16-1.pdf}

\hypertarget{making-neighbor-comparisons}{%
\subsubsection{Making neighbor
comparisons}\label{making-neighbor-comparisons}}

We can use our look-up matrix to find all of our neighbors. Here, we can
see a few examples of how this works.

\begin{Shaded}
\begin{Highlighting}[]
\NormalTok{sparse.neighbors }\OtherTok{\textless{}{-}} \FunctionTok{neighbor\_finder}\NormalTok{(m2, cell.data, neighbors.found, }\AttributeTok{quantile =} \FloatTok{0.99}\NormalTok{)}
\end{Highlighting}
\end{Shaded}

\begin{verbatim}
## [1] "Creating neighbor comparisons"
## [1] "Finding valid comparisons"
## [1] "Splitting cell.data into list"
## [1] "Using inputted list of cell data"
## [1] "Running comparisons"
\end{verbatim}

\begin{Shaded}
\begin{Highlighting}[]
\FunctionTok{par}\NormalTok{(}\AttributeTok{mfrow =} \FunctionTok{c}\NormalTok{(}\DecValTok{2}\NormalTok{,}\DecValTok{3}\NormalTok{))}
\FunctionTok{plot}\NormalTok{(}\FunctionTok{setValues}\NormalTok{(mock.surface, sparse.neighbors}\SpecialCharTok{$}\NormalTok{matrix[}\FunctionTok{sample}\NormalTok{(}\DecValTok{1}\SpecialCharTok{:}\FunctionTok{nrow}\NormalTok{(sparse.neighbors}\SpecialCharTok{$}\NormalTok{matrix), }\DecValTok{1}\NormalTok{),]))}
\FunctionTok{plot}\NormalTok{(}\FunctionTok{setValues}\NormalTok{(mock.surface, sparse.neighbors}\SpecialCharTok{$}\NormalTok{matrix[}\FunctionTok{sample}\NormalTok{(}\DecValTok{1}\SpecialCharTok{:}\FunctionTok{nrow}\NormalTok{(sparse.neighbors}\SpecialCharTok{$}\NormalTok{matrix), }\DecValTok{1}\NormalTok{),]))}
\FunctionTok{plot}\NormalTok{(}\FunctionTok{setValues}\NormalTok{(mock.surface, sparse.neighbors}\SpecialCharTok{$}\NormalTok{matrix[}\FunctionTok{sample}\NormalTok{(}\DecValTok{1}\SpecialCharTok{:}\FunctionTok{nrow}\NormalTok{(sparse.neighbors}\SpecialCharTok{$}\NormalTok{matrix), }\DecValTok{1}\NormalTok{),]))}
\FunctionTok{plot}\NormalTok{(}\FunctionTok{setValues}\NormalTok{(mock.surface, sparse.neighbors}\SpecialCharTok{$}\NormalTok{matrix[}\FunctionTok{sample}\NormalTok{(}\DecValTok{1}\SpecialCharTok{:}\FunctionTok{nrow}\NormalTok{(sparse.neighbors}\SpecialCharTok{$}\NormalTok{matrix), }\DecValTok{1}\NormalTok{),]))}
\FunctionTok{plot}\NormalTok{(}\FunctionTok{setValues}\NormalTok{(mock.surface, sparse.neighbors}\SpecialCharTok{$}\NormalTok{matrix[}\FunctionTok{sample}\NormalTok{(}\DecValTok{1}\SpecialCharTok{:}\FunctionTok{nrow}\NormalTok{(sparse.neighbors}\SpecialCharTok{$}\NormalTok{matrix), }\DecValTok{1}\NormalTok{),]))}
\FunctionTok{plot}\NormalTok{(}\FunctionTok{setValues}\NormalTok{(mock.surface, sparse.neighbors}\SpecialCharTok{$}\NormalTok{matrix[}\FunctionTok{sample}\NormalTok{(}\DecValTok{1}\SpecialCharTok{:}\FunctionTok{nrow}\NormalTok{(sparse.neighbors}\SpecialCharTok{$}\NormalTok{matrix), }\DecValTok{1}\NormalTok{),]))}
\end{Highlighting}
\end{Shaded}

\includegraphics{Untitled_files/figure-latex/unnamed-chunk-17-1.pdf}

\hypertarget{compiling-ssf-comparisons}{%
\subsubsection{Compiling SSF
comparisons}\label{compiling-ssf-comparisons}}

We can create all the comparisons between focal and non-focal cells. We
need to add some extraneous things like creating our \texttt{log\_sl}
and \texttt{cos\_ta} values. Again, log plus one for all the step
lengths to accommodate zeros. Functionally, our specification of turning
angle is BS, but it shouldn't affect inferences. In the future, turning
angle could be estimated based on the turning angle (if we wanted to get
fancy).

\begin{Shaded}
\begin{Highlighting}[]
\NormalTok{ssf.comparisons }\OtherTok{\textless{}{-}} \FunctionTok{compile\_ssf\_comparisons}\NormalTok{(sparse.neighbors, cell.data)}

\NormalTok{ssf.comparisons }\OtherTok{\textless{}{-}} \FunctionTok{lapply}\NormalTok{(ssf.comparisons, }\ControlFlowTok{function}\NormalTok{(x) \{}
\NormalTok{  x}\SpecialCharTok{$}\NormalTok{.for}\SpecialCharTok{$}\NormalTok{log\_sl }\OtherTok{\textless{}{-}} \FunctionTok{log}\NormalTok{(x}\SpecialCharTok{$}\NormalTok{.for}\SpecialCharTok{$}\NormalTok{step}\SpecialCharTok{+}\DecValTok{1}\NormalTok{)}
\NormalTok{  x}\SpecialCharTok{$}\NormalTok{.given}\SpecialCharTok{$}\NormalTok{log\_sl }\OtherTok{\textless{}{-}} \FunctionTok{log}\NormalTok{(x}\SpecialCharTok{$}\NormalTok{.given}\SpecialCharTok{$}\NormalTok{step}\SpecialCharTok{+}\DecValTok{1}\NormalTok{)}
  
\NormalTok{  x}\SpecialCharTok{$}\NormalTok{.for}\SpecialCharTok{$}\NormalTok{cos\_ta }\OtherTok{\textless{}{-}} \DecValTok{0}
\NormalTok{  x}\SpecialCharTok{$}\NormalTok{.given}\SpecialCharTok{$}\NormalTok{cos\_ta }\OtherTok{\textless{}{-}} \DecValTok{0}
  
  \FunctionTok{list}\NormalTok{(}\AttributeTok{.for =}\NormalTok{ x}\SpecialCharTok{$}\NormalTok{.for, }\AttributeTok{.given =}\NormalTok{ x}\SpecialCharTok{$}\NormalTok{.given)}
\NormalTok{\})}

\FunctionTok{head}\NormalTok{(ssf.comparisons}\SpecialCharTok{$}\StringTok{\textasciigrave{}}\AttributeTok{1}\StringTok{\textasciigrave{}}\SpecialCharTok{$}\NormalTok{.for)}
\end{Highlighting}
\end{Shaded}

\begin{verbatim}
##           x       y  ID sh.forest step_id_unique   log_sl cos_ta cellnr
## 1   4304850 3455600   1         1     step_id_=1 0.000000      0      1
## 151 4304850 3455100 151         2     step_id_=1 6.216606      0    151
## 152 4305100 3455100 152         2     step_id_=1 6.328233      0    152
## 153 4305351 3455100 153         2     step_id_=1 6.563261      0    153
## 154 4305601 3455100 154         2     step_id_=1 6.805966      0    154
## 155 4305852 3455100 155         2     step_id_=1 7.021286      0    155
##          lRSS      step
## 1   0.3440625    0.0000
## 151 0.0000000  500.0000
## 152 0.0000000  559.1661
## 153 0.0000000  707.5783
## 154 0.0000000  902.2200
## 155 0.0000000 1119.2267
\end{verbatim}

\hypertarget{predicting-surface-probabilities}{%
\subsubsection{Predicting surface
probabilities}\label{predicting-surface-probabilities}}

We can now predict our whole surface based on the transition
probabilities between each focal and each neighbor cell. We can
visualize some below.

\begin{Shaded}
\begin{Highlighting}[]
\NormalTok{surface }\OtherTok{\textless{}{-}} \FunctionTok{predict\_ssf\_comparisons}\NormalTok{(m2, ssf.comparisons)}
\end{Highlighting}
\end{Shaded}

\begin{verbatim}
## [1] "Estimating probability surface"
## [1] "Compiling probability surface"
## [1] "Making sparse matrix for transitions"
\end{verbatim}

\begin{Shaded}
\begin{Highlighting}[]
\FunctionTok{par}\NormalTok{(}\AttributeTok{mfrow =} \FunctionTok{c}\NormalTok{(}\DecValTok{3}\NormalTok{,}\DecValTok{4}\NormalTok{), }\AttributeTok{mai =} \FunctionTok{c}\NormalTok{(}\DecValTok{0}\NormalTok{, }\DecValTok{0}\NormalTok{, }\FloatTok{0.1}\NormalTok{, }\DecValTok{0}\NormalTok{))}

\FunctionTok{plot}\NormalTok{(}\FunctionTok{setValues}\NormalTok{(mock.surface, surface}\SpecialCharTok{$}\NormalTok{sparse.matrix[}\FunctionTok{sample}\NormalTok{(}\DecValTok{1}\SpecialCharTok{:}\FunctionTok{nrow}\NormalTok{(surface}\SpecialCharTok{$}\NormalTok{sparse.matrix), }\DecValTok{1}\NormalTok{),]))}
\FunctionTok{plot}\NormalTok{(}\FunctionTok{setValues}\NormalTok{(mock.surface, surface}\SpecialCharTok{$}\NormalTok{sparse.matrix[}\FunctionTok{sample}\NormalTok{(}\DecValTok{1}\SpecialCharTok{:}\FunctionTok{nrow}\NormalTok{(surface}\SpecialCharTok{$}\NormalTok{sparse.matrix), }\DecValTok{1}\NormalTok{),]))}
\FunctionTok{plot}\NormalTok{(}\FunctionTok{setValues}\NormalTok{(mock.surface, surface}\SpecialCharTok{$}\NormalTok{sparse.matrix[}\FunctionTok{sample}\NormalTok{(}\DecValTok{1}\SpecialCharTok{:}\FunctionTok{nrow}\NormalTok{(surface}\SpecialCharTok{$}\NormalTok{sparse.matrix), }\DecValTok{1}\NormalTok{),]))}
\FunctionTok{plot}\NormalTok{(}\FunctionTok{setValues}\NormalTok{(mock.surface, surface}\SpecialCharTok{$}\NormalTok{sparse.matrix[}\FunctionTok{sample}\NormalTok{(}\DecValTok{1}\SpecialCharTok{:}\FunctionTok{nrow}\NormalTok{(surface}\SpecialCharTok{$}\NormalTok{sparse.matrix), }\DecValTok{1}\NormalTok{),]))}
\FunctionTok{plot}\NormalTok{(}\FunctionTok{setValues}\NormalTok{(mock.surface, surface}\SpecialCharTok{$}\NormalTok{sparse.matrix[}\FunctionTok{sample}\NormalTok{(}\DecValTok{1}\SpecialCharTok{:}\FunctionTok{nrow}\NormalTok{(surface}\SpecialCharTok{$}\NormalTok{sparse.matrix), }\DecValTok{1}\NormalTok{),]))}
\FunctionTok{plot}\NormalTok{(}\FunctionTok{setValues}\NormalTok{(mock.surface, surface}\SpecialCharTok{$}\NormalTok{sparse.matrix[}\FunctionTok{sample}\NormalTok{(}\DecValTok{1}\SpecialCharTok{:}\FunctionTok{nrow}\NormalTok{(surface}\SpecialCharTok{$}\NormalTok{sparse.matrix), }\DecValTok{1}\NormalTok{),]))}
\FunctionTok{plot}\NormalTok{(}\FunctionTok{setValues}\NormalTok{(mock.surface, surface}\SpecialCharTok{$}\NormalTok{sparse.matrix[}\FunctionTok{sample}\NormalTok{(}\DecValTok{1}\SpecialCharTok{:}\FunctionTok{nrow}\NormalTok{(surface}\SpecialCharTok{$}\NormalTok{sparse.matrix), }\DecValTok{1}\NormalTok{),]))}
\FunctionTok{plot}\NormalTok{(}\FunctionTok{setValues}\NormalTok{(mock.surface, surface}\SpecialCharTok{$}\NormalTok{sparse.matrix[}\FunctionTok{sample}\NormalTok{(}\DecValTok{1}\SpecialCharTok{:}\FunctionTok{nrow}\NormalTok{(surface}\SpecialCharTok{$}\NormalTok{sparse.matrix), }\DecValTok{1}\NormalTok{),]))}
\FunctionTok{plot}\NormalTok{(}\FunctionTok{setValues}\NormalTok{(mock.surface, surface}\SpecialCharTok{$}\NormalTok{sparse.matrix[}\FunctionTok{sample}\NormalTok{(}\DecValTok{1}\SpecialCharTok{:}\FunctionTok{nrow}\NormalTok{(surface}\SpecialCharTok{$}\NormalTok{sparse.matrix), }\DecValTok{1}\NormalTok{),]))}
\FunctionTok{plot}\NormalTok{(}\FunctionTok{setValues}\NormalTok{(mock.surface, surface}\SpecialCharTok{$}\NormalTok{sparse.matrix[}\FunctionTok{sample}\NormalTok{(}\DecValTok{1}\SpecialCharTok{:}\FunctionTok{nrow}\NormalTok{(surface}\SpecialCharTok{$}\NormalTok{sparse.matrix), }\DecValTok{1}\NormalTok{),]))}
\FunctionTok{plot}\NormalTok{(}\FunctionTok{setValues}\NormalTok{(mock.surface, surface}\SpecialCharTok{$}\NormalTok{sparse.matrix[}\FunctionTok{sample}\NormalTok{(}\DecValTok{1}\SpecialCharTok{:}\FunctionTok{nrow}\NormalTok{(surface}\SpecialCharTok{$}\NormalTok{sparse.matrix), }\DecValTok{1}\NormalTok{),]))}
\FunctionTok{plot}\NormalTok{(}\FunctionTok{setValues}\NormalTok{(mock.surface, surface}\SpecialCharTok{$}\NormalTok{sparse.matrix[}\FunctionTok{sample}\NormalTok{(}\DecValTok{1}\SpecialCharTok{:}\FunctionTok{nrow}\NormalTok{(surface}\SpecialCharTok{$}\NormalTok{sparse.matrix), }\DecValTok{1}\NormalTok{),]))}
\end{Highlighting}
\end{Shaded}

\includegraphics{Untitled_files/figure-latex/unnamed-chunk-19-1.pdf}

\hypertarget{making-graph}{%
\subsubsection{Making graph}\label{making-graph}}

We can use these relationships to make a graph. We will trim any NA
nodes and edges, but we should record their identity for mapping back to
the prediction surface. This graph is absolutely huge, so it not really
legible and is not plotted here.

\begin{Shaded}
\begin{Highlighting}[]
\NormalTok{g }\OtherTok{\textless{}{-}} \FunctionTok{graph\_from\_adjacency\_matrix}\NormalTok{(surface}\SpecialCharTok{$}\NormalTok{sparse.matrix, }\AttributeTok{mode =} \StringTok{"directed"}\NormalTok{, }\AttributeTok{weighted =}\NormalTok{ T, }\AttributeTok{diag =}\NormalTok{ T)}
\FunctionTok{V}\NormalTok{(g)}\SpecialCharTok{$}\NormalTok{name }\OtherTok{\textless{}{-}} \FunctionTok{V}\NormalTok{(g)}
\NormalTok{Isolated }\OtherTok{\textless{}{-}} \FunctionTok{which}\NormalTok{(}\FunctionTok{degree}\NormalTok{(g)}\SpecialCharTok{==}\DecValTok{0}\NormalTok{)}
\NormalTok{Connected }\OtherTok{\textless{}{-}} \FunctionTok{which}\NormalTok{(}\FunctionTok{degree}\NormalTok{(g)}\SpecialCharTok{\textgreater{}}\DecValTok{0}\NormalTok{)}
\NormalTok{g }\OtherTok{\textless{}{-}} \FunctionTok{delete.vertices}\NormalTok{(g, Isolated)}
\NormalTok{g }\OtherTok{\textless{}{-}} \FunctionTok{delete.edges}\NormalTok{(g, }\FunctionTok{E}\NormalTok{(g)[}\FunctionTok{is.na}\NormalTok{(weight)])}
\CommentTok{\# plot(g)}
\end{Highlighting}
\end{Shaded}

\hypertarget{calculating-centrality}{%
\subsubsection{Calculating centrality}\label{calculating-centrality}}

We can calculate PageRank centrality, which is arguably the closest
thing to an RSF-like definition. As you can see, its different than the
above RSS surface.

\begin{Shaded}
\begin{Highlighting}[]
\NormalTok{pg }\OtherTok{\textless{}{-}} \FunctionTok{page\_rank}\NormalTok{(g)}
\NormalTok{page.rank.raster }\OtherTok{\textless{}{-}} \FunctionTok{setValues}\NormalTok{(mock.surface, pg}\SpecialCharTok{$}\NormalTok{vect)}
\FunctionTok{plot}\NormalTok{(page.rank.raster)}
\FunctionTok{points}\NormalTok{(deer, }\AttributeTok{pch =} \StringTok{"."}\NormalTok{, }\AttributeTok{col =} \FunctionTok{alpha}\NormalTok{(}\StringTok{"black"}\NormalTok{, }\FloatTok{0.25}\NormalTok{))}
\end{Highlighting}
\end{Shaded}

\includegraphics{Untitled_files/figure-latex/unnamed-chunk-21-1.pdf}

\hypertarget{calculating-other-information-for-comparison}{%
\subsubsection{Calculating other information for
comparison}\label{calculating-other-information-for-comparison}}

We can also calculate other things, like the sum of all probabilities
favoring a cell across all pertinent neighbors. Or the diagonal of our
transition matrix, which is the probability of selecting a focal cell
given the neighborhood.

\begin{Shaded}
\begin{Highlighting}[]
\NormalTok{diag }\OtherTok{\textless{}{-}} \FunctionTok{diag}\NormalTok{(surface}\SpecialCharTok{$}\NormalTok{sparse.matrix)}
\NormalTok{sum.prob }\OtherTok{\textless{}{-}} \FunctionTok{colSums}\NormalTok{(surface}\SpecialCharTok{$}\NormalTok{sparse.matrix)}
\NormalTok{sum.m.diag }\OtherTok{\textless{}{-}}\NormalTok{ sum.prob }\SpecialCharTok{{-}}\NormalTok{ diag}
\NormalTok{neighbors }\OtherTok{\textless{}{-}} \FunctionTok{unlist}\NormalTok{(}\FunctionTok{lapply}\NormalTok{(ssf.comparisons, }\ControlFlowTok{function}\NormalTok{(x) }\FunctionTok{nrow}\NormalTok{(x}\SpecialCharTok{$}\NormalTok{.for)))}

\FunctionTok{par}\NormalTok{(}\AttributeTok{mfrow =} \FunctionTok{c}\NormalTok{(}\DecValTok{2}\NormalTok{,}\DecValTok{2}\NormalTok{), }\AttributeTok{mai =} \FunctionTok{c}\NormalTok{(}\DecValTok{0}\NormalTok{,}\DecValTok{0}\NormalTok{,}\FloatTok{0.2}\NormalTok{,}\DecValTok{0}\NormalTok{))}
\FunctionTok{plot}\NormalTok{(}\FunctionTok{setValues}\NormalTok{(mock.surface, diag), }\AttributeTok{main =} \StringTok{"Diagonal"}\NormalTok{)}
\FunctionTok{plot}\NormalTok{(}\FunctionTok{setValues}\NormalTok{(mock.surface, sum.prob), }\AttributeTok{main =} \StringTok{"Sum"}\NormalTok{)}
\FunctionTok{plot}\NormalTok{(}\FunctionTok{setValues}\NormalTok{(mock.surface, sum.m.diag), }\AttributeTok{main =} \StringTok{"Sum {-} Diagonal"}\NormalTok{)}
\FunctionTok{plot}\NormalTok{(}\FunctionTok{setValues}\NormalTok{(mock.surface, neighbors), }\AttributeTok{main =} \StringTok{"Neighbors"}\NormalTok{)}
\end{Highlighting}
\end{Shaded}

\includegraphics{Untitled_files/figure-latex/unnamed-chunk-22-1.pdf}

\hypertarget{making-surfaces-from-predictions}{%
\subsubsection{Making surfaces from
predictions}\label{making-surfaces-from-predictions}}

Here, we can run with summed probabilities as a first proxy. We can see
there are real, tangible differences from PageRank.

\begin{Shaded}
\begin{Highlighting}[]
\NormalTok{sum.prob.raster }\OtherTok{\textless{}{-}} \FunctionTok{setValues}\NormalTok{(mock.surface, sum.prob)}

\FunctionTok{plot}\NormalTok{(spatialEco}\SpecialCharTok{::}\FunctionTok{rasterCorrelation}\NormalTok{(page.rank.raster, sum.prob.raster))}
\FunctionTok{points}\NormalTok{(deer, }\AttributeTok{pch =} \StringTok{"."}\NormalTok{, }\AttributeTok{col =} \FunctionTok{alpha}\NormalTok{(}\StringTok{"black"}\NormalTok{, }\FloatTok{0.25}\NormalTok{))}
\end{Highlighting}
\end{Shaded}

\includegraphics{Untitled_files/figure-latex/unnamed-chunk-23-1.pdf}

\hypertarget{making-a-mock-rsf}{%
\subsubsection{Making a mock RSF}\label{making-a-mock-rsf}}

We can make a mock RSF, using our deer data set in a super flawed,
overfit way. Interestingly, this RSF is PERFECTLY correlated with our
prior Log-RSS surface - so pretty cool.

\begin{Shaded}
\begin{Highlighting}[]
\NormalTok{rsf }\OtherTok{\textless{}{-}}\NormalTok{ deer }\SpecialCharTok{\%\textgreater{}\%} 
  \FunctionTok{random\_points}\NormalTok{() }\SpecialCharTok{\%\textgreater{}\%} 
  \FunctionTok{extract\_covariates}\NormalTok{(sh\_forest) }\SpecialCharTok{\%\textgreater{}\%} 
  \FunctionTok{mutate}\NormalTok{(}\AttributeTok{sh.forest =} \FunctionTok{factor}\NormalTok{(sh.forest))}

\NormalTok{rsf }\OtherTok{\textless{}{-}}\NormalTok{ rsf }\SpecialCharTok{\%\textgreater{}\%} 
  \FunctionTok{fit\_rsf}\NormalTok{(case\_ }\SpecialCharTok{\textasciitilde{}}\NormalTok{ sh.forest, }\AttributeTok{model =}\NormalTok{ T) }

\NormalTok{probabilities }\OtherTok{\textless{}{-}} \FunctionTok{exp}\NormalTok{(}\FunctionTok{predict}\NormalTok{(rsf}\SpecialCharTok{$}\NormalTok{model, }\AttributeTok{newdata =}\NormalTok{ pred.data))}\SpecialCharTok{/}\NormalTok{(}\DecValTok{1}\SpecialCharTok{+}\FunctionTok{exp}\NormalTok{(}\FunctionTok{predict}\NormalTok{(rsf}\SpecialCharTok{$}\NormalTok{model, }\AttributeTok{newdata =}\NormalTok{ pred.data)))}
\NormalTok{rsf.prob.raster }\OtherTok{\textless{}{-}} \FunctionTok{setValues}\NormalTok{(mock.surface, probabilities)}
\FunctionTok{plot}\NormalTok{(rsf.prob.raster)}
\end{Highlighting}
\end{Shaded}

\includegraphics{Untitled_files/figure-latex/unnamed-chunk-24-1.pdf}
There are, however, deviations from both the base probabilities and the
PageRank

\begin{Shaded}
\begin{Highlighting}[]
\FunctionTok{par}\NormalTok{(}\AttributeTok{mfrow =} \FunctionTok{c}\NormalTok{(}\DecValTok{1}\NormalTok{,}\DecValTok{2}\NormalTok{))}
\FunctionTok{plot}\NormalTok{(spatialEco}\SpecialCharTok{::}\FunctionTok{rasterCorrelation}\NormalTok{(page.rank.raster, rsf.prob.raster))}
\FunctionTok{points}\NormalTok{(deer, }\AttributeTok{pch =} \StringTok{"."}\NormalTok{, }\AttributeTok{col =} \FunctionTok{alpha}\NormalTok{(}\StringTok{"black"}\NormalTok{, }\FloatTok{0.25}\NormalTok{), }\AttributeTok{main =} \StringTok{"RSF vs PageRank"}\NormalTok{)}
\FunctionTok{plot}\NormalTok{(spatialEco}\SpecialCharTok{::}\FunctionTok{rasterCorrelation}\NormalTok{(sum.prob.raster, rsf.prob.raster))}
\FunctionTok{points}\NormalTok{(deer, }\AttributeTok{pch =} \StringTok{"."}\NormalTok{, }\AttributeTok{col =} \FunctionTok{alpha}\NormalTok{(}\StringTok{"black"}\NormalTok{, }\FloatTok{0.25}\NormalTok{), }\AttributeTok{main =} \StringTok{"RSF vs Probability"}\NormalTok{)}
\end{Highlighting}
\end{Shaded}

\includegraphics{Untitled_files/figure-latex/unnamed-chunk-25-1.pdf}

\hypertarget{comparing-rsf-to-ssf-surfaces}{%
\subsubsection{Comparing RSF to SSF
surfaces}\label{comparing-rsf-to-ssf-surfaces}}

We can also compare the data they contain, and what predictive power
they have for the deer locations we observed.

\begin{Shaded}
\begin{Highlighting}[]
\NormalTok{ssf1 }\OtherTok{\textless{}{-}}\NormalTok{ deer }\SpecialCharTok{\%\textgreater{}\%} 
      \FunctionTok{extract\_covariates}\NormalTok{(}\FunctionTok{stack}\NormalTok{(rsf.prob.raster, sum.prob.raster, page.rank.raster)) }

\NormalTok{quantiles.pred }\OtherTok{\textless{}{-}} \FunctionTok{quantile}\NormalTok{(}\FunctionTok{values}\NormalTok{(rsf.prob.raster), }\FunctionTok{seq}\NormalTok{(}\FloatTok{0.01}\NormalTok{, }\DecValTok{1}\NormalTok{, }\AttributeTok{by =} \FloatTok{0.01}\NormalTok{), }\AttributeTok{na.rm =}\NormalTok{ T)}
\NormalTok{percent.in.a }\OtherTok{\textless{}{-}} \FunctionTok{rep}\NormalTok{(}\DecValTok{0}\NormalTok{,}\DecValTok{100}\NormalTok{)}
\ControlFlowTok{for}\NormalTok{ (i }\ControlFlowTok{in} \DecValTok{1}\SpecialCharTok{:}\NormalTok{(}\FunctionTok{length}\NormalTok{(quantiles.pred)))\{}
\NormalTok{  yup }\OtherTok{\textless{}{-}}\NormalTok{ ssf1}\SpecialCharTok{$}\NormalTok{layer}\FloatTok{.1} \SpecialCharTok{\textless{}=}\NormalTok{ quantiles.pred[i]}
\NormalTok{  percent.in.a[i] }\OtherTok{\textless{}{-}} \FunctionTok{sum}\NormalTok{(yup, }\AttributeTok{na.rm =}\NormalTok{ T)}
\NormalTok{\}}
\NormalTok{percent.in.a }\OtherTok{\textless{}{-}}\NormalTok{ percent.in.a}\SpecialCharTok{/}\FunctionTok{table}\NormalTok{(}\FunctionTok{is.na}\NormalTok{(ssf1}\SpecialCharTok{$}\NormalTok{layer}\FloatTok{.1}\NormalTok{))[}\DecValTok{1}\NormalTok{]}

\NormalTok{quantiles.pred }\OtherTok{\textless{}{-}} \FunctionTok{quantile}\NormalTok{(}\FunctionTok{values}\NormalTok{(sum.prob.raster), }\FunctionTok{seq}\NormalTok{(}\FloatTok{0.01}\NormalTok{, }\DecValTok{1}\NormalTok{, }\AttributeTok{by =} \FloatTok{0.01}\NormalTok{), }\AttributeTok{na.rm =}\NormalTok{ T)}
\NormalTok{percent.in.b }\OtherTok{\textless{}{-}} \FunctionTok{rep}\NormalTok{(}\DecValTok{0}\NormalTok{,}\DecValTok{100}\NormalTok{)}
\ControlFlowTok{for}\NormalTok{ (i }\ControlFlowTok{in} \DecValTok{1}\SpecialCharTok{:}\NormalTok{(}\FunctionTok{length}\NormalTok{(quantiles.pred)))\{}
\NormalTok{  yup }\OtherTok{\textless{}{-}}\NormalTok{ ssf1}\SpecialCharTok{$}\NormalTok{layer}\FloatTok{.2} \SpecialCharTok{\textless{}=}\NormalTok{ quantiles.pred[i]}
\NormalTok{  percent.in.b[i] }\OtherTok{\textless{}{-}} \FunctionTok{sum}\NormalTok{(yup, }\AttributeTok{na.rm =}\NormalTok{ T)}
\NormalTok{\}}
\NormalTok{percent.in.b }\OtherTok{\textless{}{-}}\NormalTok{ percent.in.b}\SpecialCharTok{/}\FunctionTok{table}\NormalTok{(}\FunctionTok{is.na}\NormalTok{(ssf1}\SpecialCharTok{$}\NormalTok{layer}\FloatTok{.2}\NormalTok{))[}\DecValTok{1}\NormalTok{]}

\NormalTok{quantiles.pred }\OtherTok{\textless{}{-}} \FunctionTok{quantile}\NormalTok{(}\FunctionTok{values}\NormalTok{(page.rank.raster), }\FunctionTok{seq}\NormalTok{(}\FloatTok{0.01}\NormalTok{, }\DecValTok{1}\NormalTok{, }\AttributeTok{by =} \FloatTok{0.01}\NormalTok{), }\AttributeTok{na.rm =}\NormalTok{ T)}
\NormalTok{percent.in.c }\OtherTok{\textless{}{-}} \FunctionTok{rep}\NormalTok{(}\DecValTok{0}\NormalTok{,}\DecValTok{100}\NormalTok{)}
\ControlFlowTok{for}\NormalTok{ (i }\ControlFlowTok{in} \DecValTok{1}\SpecialCharTok{:}\NormalTok{(}\FunctionTok{length}\NormalTok{(quantiles.pred)))\{}
\NormalTok{  yup }\OtherTok{\textless{}{-}}\NormalTok{ ssf1}\SpecialCharTok{$}\NormalTok{layer}\FloatTok{.3} \SpecialCharTok{\textless{}=}\NormalTok{ quantiles.pred[i]}
\NormalTok{  percent.in.c[i] }\OtherTok{\textless{}{-}} \FunctionTok{sum}\NormalTok{(yup, }\AttributeTok{na.rm =}\NormalTok{ T)}
\NormalTok{\}}
\NormalTok{percent.in.c }\OtherTok{\textless{}{-}}\NormalTok{ percent.in.c}\SpecialCharTok{/}\FunctionTok{table}\NormalTok{(}\FunctionTok{is.na}\NormalTok{(ssf1}\SpecialCharTok{$}\NormalTok{layer}\FloatTok{.3}\NormalTok{))[}\DecValTok{1}\NormalTok{]}

\FunctionTok{tibble}\NormalTok{(}\AttributeTok{percent\_pred =} \FunctionTok{rep}\NormalTok{(}\FunctionTok{seq}\NormalTok{(}\DecValTok{0}\NormalTok{, }\DecValTok{1}\NormalTok{, }\AttributeTok{by =} \FloatTok{0.01}\NormalTok{)[}\SpecialCharTok{{-}}\DecValTok{1}\NormalTok{],}\DecValTok{3}\NormalTok{),}
       \AttributeTok{precent\_obs =} \FunctionTok{c}\NormalTok{(}\DecValTok{1}\SpecialCharTok{{-}}\NormalTok{percent.in.a,}
                       \DecValTok{1}\SpecialCharTok{{-}}\NormalTok{percent.in.b,}
                       \DecValTok{1}\SpecialCharTok{{-}}\NormalTok{percent.in.c),}
       \AttributeTok{pred\_type =} \FunctionTok{rep}\NormalTok{(}\FunctionTok{c}\NormalTok{(}\StringTok{"RSF"}\NormalTok{, }\StringTok{"Summed Probability iSFF"}\NormalTok{, }\StringTok{"PageRank iSFF"}\NormalTok{), }
                       \AttributeTok{each =} \DecValTok{100}\NormalTok{)) }\SpecialCharTok{\%\textgreater{}\%} 
  \FunctionTok{group\_by}\NormalTok{(pred\_type) }\SpecialCharTok{\%\textgreater{}\%} 
  \CommentTok{\# mutate(cum\_percent = cumsum(precent\_obs)) \%\textgreater{}\% }
  \FunctionTok{ggplot}\NormalTok{(}\FunctionTok{aes}\NormalTok{(}\AttributeTok{x =}\NormalTok{ percent\_pred, }\AttributeTok{y =}\NormalTok{ precent\_obs, }\AttributeTok{color =}\NormalTok{ pred\_type)) }\SpecialCharTok{+}
  \FunctionTok{geom\_point}\NormalTok{() }\SpecialCharTok{+}
  \FunctionTok{geom\_line}\NormalTok{() }\SpecialCharTok{+}
  \FunctionTok{theme\_minimal}\NormalTok{() }\SpecialCharTok{+}
  \FunctionTok{labs}\NormalTok{(}\AttributeTok{x =} \StringTok{"Percentile of Predicted Surface"}\NormalTok{,}
       \AttributeTok{y =} \StringTok{"Cumulative Percent of Observed Data by Percentile"}\NormalTok{)}
\end{Highlighting}
\end{Shaded}

\includegraphics{Untitled_files/figure-latex/unnamed-chunk-26-1.pdf}

\begin{Shaded}
\begin{Highlighting}[]
\FunctionTok{plot}\NormalTok{(page.rank.raster}\SpecialCharTok{\textgreater{}}\NormalTok{quantiles.pred[}\FunctionTok{min}\NormalTok{(}\FunctionTok{which}\NormalTok{((}\DecValTok{1}\SpecialCharTok{{-}}\NormalTok{percent.in.c) }\SpecialCharTok{\textless{}} \FloatTok{0.9}\NormalTok{))])}
\FunctionTok{points}\NormalTok{(amt\_fisher, }\AttributeTok{pch =} \StringTok{"."}\NormalTok{)}
\end{Highlighting}
\end{Shaded}

\includegraphics{Untitled_files/figure-latex/unnamed-chunk-27-1.pdf}

\hypertarget{fishers}{%
\subsection{Fishers}\label{fishers}}

This is data for a few fishers with elevation, land cover, and
population density. I added aspect, TRI, and slope for fun. All the
subheadings are largely the same, so I won't go into too much detail.

\hypertarget{data-1}{%
\subsubsection{Data}\label{data-1}}

I created some new variables, and because extents are all different,
they have to be called separately for pulling data. This is relevant
later.

\begin{Shaded}
\begin{Highlighting}[]
\FunctionTok{data}\NormalTok{(}\StringTok{"amt\_fisher"}\NormalTok{)}
\FunctionTok{data}\NormalTok{(}\StringTok{"amt\_fisher\_covar"}\NormalTok{)}

\NormalTok{amt\_fisher\_covar}\SpecialCharTok{$}\NormalTok{slope }\OtherTok{\textless{}{-}} \FunctionTok{terrain}\NormalTok{(amt\_fisher\_covar}\SpecialCharTok{$}\NormalTok{elevation, }\AttributeTok{opt=}\StringTok{"slope"}\NormalTok{, }\AttributeTok{neighbors=}\DecValTok{8}\NormalTok{, }\AttributeTok{unit=}\StringTok{"degrees"}\NormalTok{)  }

\NormalTok{amt\_fisher\_covar}\SpecialCharTok{$}\NormalTok{aspect }\OtherTok{\textless{}{-}} \FunctionTok{terrain}\NormalTok{(amt\_fisher\_covar}\SpecialCharTok{$}\NormalTok{elevation, }\AttributeTok{opt=}\StringTok{"aspect"}\NormalTok{, }\AttributeTok{neighbors=}\DecValTok{8}\NormalTok{, }\AttributeTok{unit=}\StringTok{"degrees"}\NormalTok{) }

\NormalTok{amt\_fisher\_covar}\SpecialCharTok{$}\NormalTok{TRI }\OtherTok{\textless{}{-}} \FunctionTok{terrain}\NormalTok{(amt\_fisher\_covar}\SpecialCharTok{$}\NormalTok{elevation, }\AttributeTok{opt=}\StringTok{"TRI"}\NormalTok{, }\AttributeTok{neighbors=}\DecValTok{8}\NormalTok{, }\AttributeTok{unit=}\StringTok{"degrees"}\NormalTok{) }

\NormalTok{ssf1 }\OtherTok{\textless{}{-}}\NormalTok{ amt\_fisher }\SpecialCharTok{\%\textgreater{}\%} 
  \FunctionTok{filter}\NormalTok{(name }\SpecialCharTok{==} \StringTok{"Leroy"}\NormalTok{) }\SpecialCharTok{\%\textgreater{}\%} 
  \FunctionTok{track\_resample}\NormalTok{(}\AttributeTok{rate =} \FunctionTok{minutes}\NormalTok{(}\DecValTok{15}\NormalTok{), }\AttributeTok{tolerance =} \FunctionTok{minutes}\NormalTok{(}\DecValTok{3}\NormalTok{)) }\SpecialCharTok{\%\textgreater{}\%} 
  \FunctionTok{filter\_min\_n\_burst}\NormalTok{(}\AttributeTok{min\_n =} \DecValTok{3}\NormalTok{) }\SpecialCharTok{\%\textgreater{}\%} 
  \FunctionTok{steps\_by\_burst}\NormalTok{() }\SpecialCharTok{\%\textgreater{}\%} 
  \FunctionTok{random\_steps}\NormalTok{() }\SpecialCharTok{\%\textgreater{}\%} 
  \FunctionTok{extract\_covariates}\NormalTok{(amt\_fisher\_covar}\SpecialCharTok{$}\NormalTok{landuse) }\SpecialCharTok{\%\textgreater{}\%} 
  \FunctionTok{extract\_covariates}\NormalTok{(amt\_fisher\_covar}\SpecialCharTok{$}\NormalTok{elevation) }\SpecialCharTok{\%\textgreater{}\%} 
  \FunctionTok{extract\_covariates}\NormalTok{(amt\_fisher\_covar}\SpecialCharTok{$}\NormalTok{slope) }\SpecialCharTok{\%\textgreater{}\%} 
  \FunctionTok{extract\_covariates}\NormalTok{(amt\_fisher\_covar}\SpecialCharTok{$}\NormalTok{aspect) }\SpecialCharTok{\%\textgreater{}\%} 
  \FunctionTok{extract\_covariates}\NormalTok{(amt\_fisher\_covar}\SpecialCharTok{$}\NormalTok{TRI) }\SpecialCharTok{\%\textgreater{}\%} 
  \FunctionTok{extract\_covariates}\NormalTok{(amt\_fisher\_covar}\SpecialCharTok{$}\NormalTok{popden) }\SpecialCharTok{\%\textgreater{}\%} 
  \FunctionTok{mutate}\NormalTok{(}\AttributeTok{landC =} \FunctionTok{factor}\NormalTok{(landuse),}
    \CommentTok{\#        case\_when(}
    \CommentTok{\# landuse \%in\% c(81, 82) \textasciitilde{} "agri",}
    \CommentTok{\# landuse \%in\% c(41, 42, 43) \textasciitilde{} "forest",}
    \CommentTok{\# landuse \%in\% c(52) \textasciitilde{} "shrub",}
    \CommentTok{\# landuse \%in\% c(31, 71) \textasciitilde{} "grass",}
    \CommentTok{\# landuse \%in\% c(90,95) \textasciitilde{} "wet",}
    \CommentTok{\# landuse \%in\% c(11, 21, 22, 23, 24) \textasciitilde{} "other"),}
         \AttributeTok{cos\_ta =} \FunctionTok{cos}\NormalTok{(ta\_),}
         \AttributeTok{log\_sl =} \FunctionTok{log}\NormalTok{(sl\_}\SpecialCharTok{+}\DecValTok{1}\NormalTok{),}
    \AttributeTok{forest =} \FunctionTok{factor}\NormalTok{(landC }\SpecialCharTok{==} \DecValTok{50}\NormalTok{))}
\end{Highlighting}
\end{Shaded}

\hypertarget{model-1}{%
\subsubsection{Model}\label{model-1}}

\begin{Shaded}
\begin{Highlighting}[]
\NormalTok{m2 }\OtherTok{\textless{}{-}}\NormalTok{ ssf1 }\SpecialCharTok{\%\textgreater{}\%}
  \FunctionTok{fit\_clogit}\NormalTok{(case\_ }\SpecialCharTok{\textasciitilde{}}\NormalTok{ (elevation }\SpecialCharTok{+}\NormalTok{ tri }\SpecialCharTok{+}\NormalTok{ popden }\SpecialCharTok{+}\NormalTok{ slope }\SpecialCharTok{+}\NormalTok{ aspect }\SpecialCharTok{+}\NormalTok{ cos\_ta }\SpecialCharTok{+}\NormalTok{ log\_sl)}\SpecialCharTok{\^{}}\DecValTok{2} \SpecialCharTok{+} \FunctionTok{strata}\NormalTok{(step\_id\_), }\AttributeTok{model =}\NormalTok{ T)}
\FunctionTok{summary}\NormalTok{(m2)}
\end{Highlighting}
\end{Shaded}

\begin{verbatim}
## Call:
## coxph(formula = Surv(rep(1, 8976L), case_) ~ (elevation + tri + 
##     popden + slope + aspect + cos_ta + log_sl)^2 + strata(step_id_), 
##     data = data, model = ..1, method = "exact")
## 
##   n= 8908, number of events= 748 
##    (68 observations deleted due to missingness)
## 
##                        coef  exp(coef)   se(coef)      z Pr(>|z|)    
## elevation         1.725e-01  1.188e+00  3.495e-02  4.935    8e-07 ***
## tri               1.309e+00  3.703e+00  5.732e-01  2.284  0.02236 *  
## popden            1.061e-02  1.011e+00  3.704e-03  2.865  0.00417 ** 
## slope             1.017e-01  1.107e+00  5.121e-01  0.199  0.84260    
## aspect            1.300e-02  1.013e+00  7.837e-03  1.659  0.09717 .  
## cos_ta            6.910e-01  1.996e+00  1.035e+00  0.668  0.50435    
## log_sl            1.158e+00  3.185e+00  4.231e-01  2.738  0.00618 ** 
## elevation:tri    -1.112e-02  9.889e-01  4.883e-03 -2.277  0.02279 *  
## elevation:popden -9.097e-05  9.999e-01  3.742e-05 -2.431  0.01505 *  
## elevation:slope  -1.794e-03  9.982e-01  4.749e-03 -0.378  0.70564    
## elevation:aspect -8.209e-05  9.999e-01  7.874e-05 -1.043  0.29715    
## elevation:cos_ta -1.625e-02  9.839e-01  1.020e-02 -1.594  0.11101    
## elevation:log_sl -1.083e-02  9.892e-01  4.262e-03 -2.540  0.01107 *  
## tri:popden       -3.153e-04  9.997e-01  2.304e-04 -1.369  0.17111    
## tri:slope         3.270e-02  1.033e+00  1.513e-02  2.161  0.03068 *  
## tri:aspect       -3.154e-04  9.997e-01  5.296e-04 -0.596  0.55150    
## tri:cos_ta       -5.046e-02  9.508e-01  6.076e-02 -0.831  0.40624    
## tri:log_sl        6.309e-04  1.001e+00  2.647e-02  0.024  0.98099    
## popden:slope     -2.262e-04  9.998e-01  1.724e-04 -1.312  0.18964    
## popden:aspect    -2.298e-06  1.000e+00  1.584e-06 -1.451  0.14681    
## popden:cos_ta    -2.547e-04  9.997e-01  2.149e-04 -1.185  0.23586    
## popden:log_sl    -4.098e-05  1.000e+00  8.894e-05 -0.461  0.64499    
## slope:aspect     -6.380e-04  9.994e-01  3.617e-04 -1.764  0.07777 .  
## slope:cos_ta      6.175e-03  1.006e+00  4.279e-02  0.144  0.88524    
## slope:log_sl      8.053e-03  1.008e+00  1.928e-02  0.418  0.67623    
## aspect:cos_ta    -5.563e-04  9.994e-01  5.965e-04 -0.933  0.35101    
## aspect:log_sl    -3.282e-04  9.997e-01  2.595e-04 -1.265  0.20592    
## cos_ta:log_sl     2.705e-01  1.311e+00  3.203e-02  8.445  < 2e-16 ***
## ---
## Signif. codes:  0 '***' 0.001 '**' 0.01 '*' 0.05 '.' 0.1 ' ' 1
## 
##                  exp(coef) exp(-coef) lower .95 upper .95
## elevation           1.1883     0.8416    1.1096    1.2725
## tri                 3.7030     0.2700    1.2042   11.3874
## popden              1.0107     0.9894    1.0034    1.0180
## slope               1.1070     0.9033    0.4058    3.0201
## aspect              1.0131     0.9871    0.9976    1.0288
## cos_ta              1.9957     0.5011    0.2625   15.1737
## log_sl              3.1850     0.3140    1.3898    7.2991
## elevation:tri       0.9889     1.0112    0.9795    0.9985
## elevation:popden    0.9999     1.0001    0.9998    1.0000
## elevation:slope     0.9982     1.0018    0.9890    1.0075
## elevation:aspect    0.9999     1.0001    0.9998    1.0001
## elevation:cos_ta    0.9839     1.0164    0.9644    1.0037
## elevation:log_sl    0.9892     1.0109    0.9810    0.9975
## tri:popden          0.9997     1.0003    0.9992    1.0001
## tri:slope           1.0332     0.9678    1.0030    1.0643
## tri:aspect          0.9997     1.0003    0.9986    1.0007
## tri:cos_ta          0.9508     1.0518    0.8440    1.0710
## tri:log_sl          1.0006     0.9994    0.9500    1.0539
## popden:slope        0.9998     1.0002    0.9994    1.0001
## popden:aspect       1.0000     1.0000    1.0000    1.0000
## popden:cos_ta       0.9997     1.0003    0.9993    1.0002
## popden:log_sl       1.0000     1.0000    0.9998    1.0001
## slope:aspect        0.9994     1.0006    0.9987    1.0001
## slope:cos_ta        1.0062     0.9938    0.9253    1.0942
## slope:log_sl        1.0081     0.9920    0.9707    1.0469
## aspect:cos_ta       0.9994     1.0006    0.9983    1.0006
## aspect:log_sl       0.9997     1.0003    0.9992    1.0002
## cos_ta:log_sl       1.3106     0.7630    1.2309    1.3956
## 
## Concordance= 0.669  (se = 0.012 )
## Likelihood ratio test= 227.4  on 28 df,   p=<2e-16
## Wald test            = 211.2  on 28 df,   p=<2e-16
## Score (logrank) test = 230.2  on 28 df,   p=<2e-16
\end{verbatim}

\hypertarget{surface-1}{%
\subsubsection{Surface}\label{surface-1}}

\begin{Shaded}
\begin{Highlighting}[]
\NormalTok{mock.surface }\OtherTok{\textless{}{-}} \FunctionTok{create\_mock\_surface}\NormalTok{(amt\_fisher\_covar, T, }\FunctionTok{list}\NormalTok{(}\AttributeTok{x =} \DecValTok{200}\NormalTok{, }\AttributeTok{y =} \DecValTok{200}\NormalTok{))}
\end{Highlighting}
\end{Shaded}

\hypertarget{getting-cell-data-1}{%
\subsubsection{Getting cell data}\label{getting-cell-data-1}}

Because we cant pull all underlying raster data in one pull, we have to
do it multiple times for each variable and then merge. Then we have to
sort by cell number to get things square.

\begin{Shaded}
\begin{Highlighting}[]
\NormalTok{pred.data.l }\OtherTok{\textless{}{-}} \FunctionTok{get\_cells}\NormalTok{(m2, }
\NormalTok{                       mock.surface,}
\NormalTok{                       amt\_fisher\_covar}\SpecialCharTok{$}\NormalTok{landuse, }
                       \AttributeTok{accessory.x.preds =} \FunctionTok{list}\NormalTok{(}\AttributeTok{log\_sl =} \FunctionTok{log}\NormalTok{(}\FunctionTok{step\_distance}\NormalTok{(m2, }\FloatTok{0.5}\NormalTok{)), }
                                                \AttributeTok{cos\_ta =} \DecValTok{1}\NormalTok{))}
\NormalTok{pred.data.e }\OtherTok{\textless{}{-}} \FunctionTok{get\_cells}\NormalTok{(m2, }
\NormalTok{                       mock.surface,}
\NormalTok{                       amt\_fisher\_covar}\SpecialCharTok{$}\NormalTok{elevation, }
                       \AttributeTok{accessory.x.preds =} \FunctionTok{list}\NormalTok{(}\AttributeTok{log\_sl =} \FunctionTok{log}\NormalTok{(}\FunctionTok{step\_distance}\NormalTok{(m2, }\FloatTok{0.5}\NormalTok{)), }
                                                \AttributeTok{cos\_ta =} \DecValTok{1}\NormalTok{))}
\NormalTok{pred.data.p }\OtherTok{\textless{}{-}} \FunctionTok{get\_cells}\NormalTok{(m2, }
\NormalTok{                       mock.surface,}
\NormalTok{                       amt\_fisher\_covar}\SpecialCharTok{$}\NormalTok{popden, }
                       \AttributeTok{accessory.x.preds =} \FunctionTok{list}\NormalTok{(}\AttributeTok{log\_sl =} \FunctionTok{log}\NormalTok{(}\FunctionTok{step\_distance}\NormalTok{(m2, }\FloatTok{0.5}\NormalTok{)), }
                                                \AttributeTok{cos\_ta =} \DecValTok{1}\NormalTok{))}

\NormalTok{pred.data.s }\OtherTok{\textless{}{-}} \FunctionTok{get\_cells}\NormalTok{(m2, }
\NormalTok{                       mock.surface,}
\NormalTok{                       amt\_fisher\_covar}\SpecialCharTok{$}\NormalTok{slope, }
                       \AttributeTok{accessory.x.preds =} \FunctionTok{list}\NormalTok{(}\AttributeTok{log\_sl =} \FunctionTok{log}\NormalTok{(}\FunctionTok{step\_distance}\NormalTok{(m2, }\FloatTok{0.5}\NormalTok{)), }
                                                \AttributeTok{cos\_ta =} \DecValTok{1}\NormalTok{))}

\NormalTok{pred.data.a }\OtherTok{\textless{}{-}} \FunctionTok{get\_cells}\NormalTok{(m2, }
\NormalTok{                       mock.surface,}
\NormalTok{                       amt\_fisher\_covar}\SpecialCharTok{$}\NormalTok{aspect, }
                       \AttributeTok{accessory.x.preds =} \FunctionTok{list}\NormalTok{(}\AttributeTok{log\_sl =} \FunctionTok{log}\NormalTok{(}\FunctionTok{step\_distance}\NormalTok{(m2, }\FloatTok{0.5}\NormalTok{)), }
                                                \AttributeTok{cos\_ta =} \DecValTok{1}\NormalTok{))}

\NormalTok{pred.data.t }\OtherTok{\textless{}{-}} \FunctionTok{get\_cells}\NormalTok{(m2, }
\NormalTok{                       mock.surface,}
\NormalTok{                       amt\_fisher\_covar}\SpecialCharTok{$}\NormalTok{TRI, }
                       \AttributeTok{accessory.x.preds =} \FunctionTok{list}\NormalTok{(}\AttributeTok{log\_sl =} \FunctionTok{log}\NormalTok{(}\FunctionTok{step\_distance}\NormalTok{(m2, }\FloatTok{0.5}\NormalTok{)), }
                                                \AttributeTok{cos\_ta =} \DecValTok{1}\NormalTok{))}

\NormalTok{pred.data }\OtherTok{\textless{}{-}} \FunctionTok{merge}\NormalTok{(}
  \FunctionTok{merge}\NormalTok{(}
    \FunctionTok{merge}\NormalTok{(}
      \FunctionTok{merge}\NormalTok{(}
        \FunctionTok{merge}\NormalTok{(pred.data.l,pred.data.e),}
\NormalTok{        pred.data.p),}
\NormalTok{      pred.data.s),}
\NormalTok{    pred.data.a),}
\NormalTok{  pred.data.t)}

\NormalTok{pred.data }\OtherTok{\textless{}{-}}\NormalTok{ pred.data }\SpecialCharTok{\%\textgreater{}\%} 
  \FunctionTok{arrange}\NormalTok{(cellnr)}

\NormalTok{pred.data}\SpecialCharTok{$}\NormalTok{forest }\OtherTok{\textless{}{-}} \FunctionTok{factor}\NormalTok{(pred.data}\SpecialCharTok{$}\NormalTok{landuse }\SpecialCharTok{==} \DecValTok{50}\NormalTok{)}
\end{Highlighting}
\end{Shaded}

\hypertarget{testing-surface-1}{%
\subsubsection{Testing surface}\label{testing-surface-1}}

\begin{Shaded}
\begin{Highlighting}[]
\NormalTok{cell.data }\OtherTok{\textless{}{-}} \FunctionTok{get\_cell\_data}\NormalTok{(m2, pred.data)}
\end{Highlighting}
\end{Shaded}

\hypertarget{finding-neighbors-1}{%
\subsubsection{Finding neighbors}\label{finding-neighbors-1}}

\begin{Shaded}
\begin{Highlighting}[]
\NormalTok{neighbors.found }\OtherTok{\textless{}{-}} \FunctionTok{neighbor\_lookup}\NormalTok{(mock.surface, cell.data)}
\end{Highlighting}
\end{Shaded}

\begin{verbatim}
## [1] "Splitting cell.data into list"
## [1] "Using inputted list of cell data"
\end{verbatim}

\hypertarget{making-neighbor-comparisons-1}{%
\subsubsection{Making neighbor
comparisons}\label{making-neighbor-comparisons-1}}

\begin{Shaded}
\begin{Highlighting}[]
\NormalTok{sparse.neighbors }\OtherTok{\textless{}{-}} \FunctionTok{neighbor\_finder}\NormalTok{(m2, cell.data, neighbors.found, }\AttributeTok{quantile =} \FloatTok{0.99}\NormalTok{)}
\end{Highlighting}
\end{Shaded}

\begin{verbatim}
## [1] "Creating neighbor comparisons"
## [1] "Finding valid comparisons"
## [1] "Splitting cell.data into list"
## [1] "Using inputted list of cell data"
## [1] "Running comparisons"
\end{verbatim}

\hypertarget{compiling-ssf-comparisons-1}{%
\subsubsection{Compiling SSF
comparisons}\label{compiling-ssf-comparisons-1}}

\begin{Shaded}
\begin{Highlighting}[]
\NormalTok{ssf.comparisons }\OtherTok{\textless{}{-}} \FunctionTok{compile\_ssf\_comparisons}\NormalTok{(sparse.neighbors, cell.data)}

\NormalTok{ssf.comparisons }\OtherTok{\textless{}{-}} \FunctionTok{lapply}\NormalTok{(ssf.comparisons, }\ControlFlowTok{function}\NormalTok{(x) \{}
\NormalTok{  x}\SpecialCharTok{$}\NormalTok{.for}\SpecialCharTok{$}\NormalTok{log\_sl }\OtherTok{\textless{}{-}} \FunctionTok{log}\NormalTok{(x}\SpecialCharTok{$}\NormalTok{.for}\SpecialCharTok{$}\NormalTok{step}\SpecialCharTok{+}\DecValTok{1}\NormalTok{)}
\NormalTok{  x}\SpecialCharTok{$}\NormalTok{.given}\SpecialCharTok{$}\NormalTok{log\_sl }\OtherTok{\textless{}{-}} \FunctionTok{log}\NormalTok{(x}\SpecialCharTok{$}\NormalTok{.given}\SpecialCharTok{$}\NormalTok{step}\SpecialCharTok{+}\DecValTok{1}\NormalTok{)}
  
\NormalTok{  x}\SpecialCharTok{$}\NormalTok{.for}\SpecialCharTok{$}\NormalTok{cos\_ta }\OtherTok{\textless{}{-}} \DecValTok{0}
\NormalTok{  x}\SpecialCharTok{$}\NormalTok{.given}\SpecialCharTok{$}\NormalTok{cos\_ta }\OtherTok{\textless{}{-}} \DecValTok{0}
  
  \FunctionTok{list}\NormalTok{(}\AttributeTok{.for =}\NormalTok{ x}\SpecialCharTok{$}\NormalTok{.for, }\AttributeTok{.given =}\NormalTok{ x}\SpecialCharTok{$}\NormalTok{.given)}
\NormalTok{\})}
\end{Highlighting}
\end{Shaded}

\hypertarget{predicting-surface-probabilities-1}{%
\subsubsection{Predicting surface
probabilities}\label{predicting-surface-probabilities-1}}

\begin{Shaded}
\begin{Highlighting}[]
\NormalTok{surface }\OtherTok{\textless{}{-}} \FunctionTok{predict\_ssf\_comparisons}\NormalTok{(m2, ssf.comparisons)}
\end{Highlighting}
\end{Shaded}

\begin{verbatim}
## [1] "Estimating probability surface"
## [1] "Compiling probability surface"
## [1] "Making sparse matrix for transitions"
\end{verbatim}

\hypertarget{making-graph-1}{%
\subsubsection{Making graph}\label{making-graph-1}}

\begin{Shaded}
\begin{Highlighting}[]
\NormalTok{g }\OtherTok{\textless{}{-}} \FunctionTok{graph\_from\_adjacency\_matrix}\NormalTok{(surface}\SpecialCharTok{$}\NormalTok{sparse.matrix, }\AttributeTok{mode =} \StringTok{"directed"}\NormalTok{, }\AttributeTok{weighted =}\NormalTok{ T, }\AttributeTok{diag =}\NormalTok{ T)}
\FunctionTok{V}\NormalTok{(g)}\SpecialCharTok{$}\NormalTok{name }\OtherTok{\textless{}{-}} \FunctionTok{V}\NormalTok{(g)}
\NormalTok{Isolated }\OtherTok{\textless{}{-}} \FunctionTok{which}\NormalTok{(}\FunctionTok{degree}\NormalTok{(g)}\SpecialCharTok{==}\DecValTok{0}\NormalTok{)}
\NormalTok{Connected }\OtherTok{\textless{}{-}} \FunctionTok{which}\NormalTok{(}\FunctionTok{degree}\NormalTok{(g)}\SpecialCharTok{\textgreater{}}\DecValTok{0}\NormalTok{)}
\NormalTok{g }\OtherTok{\textless{}{-}} \FunctionTok{delete.vertices}\NormalTok{(g, Isolated)}
\NormalTok{g }\OtherTok{\textless{}{-}} \FunctionTok{delete.edges}\NormalTok{(g, }\FunctionTok{E}\NormalTok{(g)[}\FunctionTok{is.na}\NormalTok{(weight)])}
\end{Highlighting}
\end{Shaded}

\hypertarget{calculating-centrality-1}{%
\subsubsection{Calculating centrality}\label{calculating-centrality-1}}

\begin{Shaded}
\begin{Highlighting}[]
\NormalTok{pg }\OtherTok{\textless{}{-}} \FunctionTok{page\_rank}\NormalTok{(g)}

\NormalTok{values }\OtherTok{\textless{}{-}} \FunctionTok{rep}\NormalTok{(}\ConstantTok{NA}\NormalTok{, }\FunctionTok{length}\NormalTok{(mock.surface))}
\NormalTok{values[}\FunctionTok{V}\NormalTok{(g)}\SpecialCharTok{$}\NormalTok{name] }\OtherTok{\textless{}{-}}\NormalTok{ pg}\SpecialCharTok{$}\NormalTok{vector}
\NormalTok{page.rank.raster }\OtherTok{\textless{}{-}} \FunctionTok{setValues}\NormalTok{(mock.surface, values)}

\FunctionTok{plot}\NormalTok{(page.rank.raster, }\AttributeTok{interpolate =}\NormalTok{ F)}
\FunctionTok{points}\NormalTok{(amt\_fisher }\SpecialCharTok{\%\textgreater{}\%} \FunctionTok{filter}\NormalTok{(name }\SpecialCharTok{==} \StringTok{"Leroy"}\NormalTok{), }\AttributeTok{pch =} \StringTok{"."}\NormalTok{, }\AttributeTok{col =} \FunctionTok{alpha}\NormalTok{(}\StringTok{"black"}\NormalTok{, }\FloatTok{0.1}\NormalTok{))}
\end{Highlighting}
\end{Shaded}

\includegraphics{Untitled_files/figure-latex/unnamed-chunk-38-1.pdf}

\hypertarget{making-a-mock-rsf-1}{%
\subsubsection{Making a mock RSF}\label{making-a-mock-rsf-1}}

\begin{Shaded}
\begin{Highlighting}[]
\NormalTok{rsf.data }\OtherTok{\textless{}{-}}\NormalTok{ amt\_fisher }\SpecialCharTok{\%\textgreater{}\%} 
  \FunctionTok{filter}\NormalTok{(name }\SpecialCharTok{==} \StringTok{"Leroy"}\NormalTok{) }\SpecialCharTok{\%\textgreater{}\%} 
  \FunctionTok{random\_points}\NormalTok{() }\SpecialCharTok{\%\textgreater{}\%} 
  \FunctionTok{extract\_covariates}\NormalTok{(amt\_fisher\_covar}\SpecialCharTok{$}\NormalTok{landuse) }\SpecialCharTok{\%\textgreater{}\%} 
  \FunctionTok{extract\_covariates}\NormalTok{(amt\_fisher\_covar}\SpecialCharTok{$}\NormalTok{elevation) }\SpecialCharTok{\%\textgreater{}\%} 
  \FunctionTok{extract\_covariates}\NormalTok{(amt\_fisher\_covar}\SpecialCharTok{$}\NormalTok{slope) }\SpecialCharTok{\%\textgreater{}\%} 
  \FunctionTok{extract\_covariates}\NormalTok{(amt\_fisher\_covar}\SpecialCharTok{$}\NormalTok{aspect) }\SpecialCharTok{\%\textgreater{}\%} 
  \FunctionTok{extract\_covariates}\NormalTok{(amt\_fisher\_covar}\SpecialCharTok{$}\NormalTok{TRI) }\SpecialCharTok{\%\textgreater{}\%} 
  \FunctionTok{extract\_covariates}\NormalTok{(amt\_fisher\_covar}\SpecialCharTok{$}\NormalTok{popden) }\SpecialCharTok{\%\textgreater{}\%} 
  \FunctionTok{mutate}\NormalTok{(}\AttributeTok{landC =} \FunctionTok{factor}\NormalTok{(landuse),}
    \AttributeTok{forest =} \FunctionTok{factor}\NormalTok{(landC }\SpecialCharTok{==} \DecValTok{50}\NormalTok{))}

\NormalTok{rsf }\OtherTok{\textless{}{-}}\NormalTok{ rsf.data }\SpecialCharTok{\%\textgreater{}\%} 
  \FunctionTok{fit\_rsf}\NormalTok{(case\_ }\SpecialCharTok{\textasciitilde{}}\NormalTok{ (elevation }\SpecialCharTok{+}\NormalTok{ tri }\SpecialCharTok{+}\NormalTok{ popden }\SpecialCharTok{+}\NormalTok{ slope }\SpecialCharTok{+}\NormalTok{ aspect)}\SpecialCharTok{\^{}}\DecValTok{2}\NormalTok{, }\AttributeTok{model =}\NormalTok{ T) }
\FunctionTok{summary}\NormalTok{(rsf)}
\end{Highlighting}
\end{Shaded}

\begin{verbatim}
## 
## Call:
## stats::glm(formula = formula, family = stats::binomial(link = "logit"), 
##     data = data, model = ..1)
## 
## Deviance Residuals: 
##     Min       1Q   Median       3Q      Max  
## -1.8255  -0.4441  -0.3749  -0.3160   2.7220  
## 
## Coefficients:
##                    Estimate Std. Error z value Pr(>|z|)    
## (Intercept)      -1.131e+01  2.234e+00  -5.060 4.19e-07 ***
## elevation         7.129e-02  2.261e-02   3.153 0.001619 ** 
## tri               3.800e-01  5.178e-01   0.734 0.463013    
## popden            1.582e-02  2.437e-03   6.492 8.47e-11 ***
## slope             3.425e-01  4.097e-01   0.836 0.403086    
## aspect            2.610e-03  6.386e-03   0.409 0.682790    
## elevation:tri     1.277e-03  4.812e-03   0.265 0.790706    
## elevation:popden -1.343e-04  2.520e-05  -5.328 9.91e-08 ***
## elevation:slope  -2.640e-03  3.928e-03  -0.672 0.501483    
## elevation:aspect  1.172e-05  6.546e-05   0.179 0.857912    
## tri:popden       -4.813e-04  1.972e-04  -2.440 0.014684 *  
## tri:slope         1.170e-03  1.204e-02   0.097 0.922606    
## tri:aspect       -4.346e-04  4.077e-04  -1.066 0.286487    
## popden:slope     -2.143e-04  1.488e-04  -1.441 0.149701    
## popden:aspect     8.870e-07  1.322e-06   0.671 0.502306    
## slope:aspect     -1.027e-03  2.918e-04  -3.521 0.000431 ***
## ---
## Signif. codes:  0 '***' 0.001 '**' 0.01 '*' 0.05 '.' 0.1 ' ' 1
## 
## (Dispersion parameter for binomial family taken to be 1)
## 
##     Null deviance: 6159.1  on 10108  degrees of freedom
## Residual deviance: 5755.4  on 10093  degrees of freedom
## AIC: 5787.4
## 
## Number of Fisher Scoring iterations: 5
\end{verbatim}

\begin{Shaded}
\begin{Highlighting}[]
\NormalTok{probabilities }\OtherTok{\textless{}{-}} \FunctionTok{exp}\NormalTok{(}\FunctionTok{predict}\NormalTok{(rsf}\SpecialCharTok{$}\NormalTok{model, }\AttributeTok{newdata =}\NormalTok{ pred.data))}\SpecialCharTok{/}\NormalTok{(}\DecValTok{1}\SpecialCharTok{+}\FunctionTok{exp}\NormalTok{(}\FunctionTok{predict}\NormalTok{(rsf}\SpecialCharTok{$}\NormalTok{model, }\AttributeTok{newdata =}\NormalTok{ pred.data)))}
\NormalTok{rsf.prob.raster }\OtherTok{\textless{}{-}} \FunctionTok{setValues}\NormalTok{(mock.surface, probabilities)}
\FunctionTok{plot}\NormalTok{(rsf.prob.raster)}
\FunctionTok{points}\NormalTok{(amt\_fisher }\SpecialCharTok{\%\textgreater{}\%} \FunctionTok{filter}\NormalTok{(name }\SpecialCharTok{==} \StringTok{"Leroy"}\NormalTok{), }\AttributeTok{pch =} \StringTok{"."}\NormalTok{, }\AttributeTok{col =} \FunctionTok{alpha}\NormalTok{(}\StringTok{"black"}\NormalTok{, }\FloatTok{0.25}\NormalTok{))}
\end{Highlighting}
\end{Shaded}

\includegraphics{Untitled_files/figure-latex/unnamed-chunk-39-1.pdf}

Lots of interesting differences from the RSF.

\begin{Shaded}
\begin{Highlighting}[]
\FunctionTok{plot}\NormalTok{(spatialEco}\SpecialCharTok{::}\FunctionTok{rasterCorrelation}\NormalTok{(page.rank.raster, rsf.prob.raster))}
\FunctionTok{points}\NormalTok{(deer, }\AttributeTok{pch =} \StringTok{"."}\NormalTok{, }\AttributeTok{col =} \FunctionTok{alpha}\NormalTok{(}\StringTok{"black"}\NormalTok{, }\FloatTok{0.25}\NormalTok{))}
\end{Highlighting}
\end{Shaded}

\includegraphics{Untitled_files/figure-latex/unnamed-chunk-40-1.pdf}

\hypertarget{comparing-rsf-to-ssf-surfaces-1}{%
\subsubsection{Comparing RSF to SSF
surfaces}\label{comparing-rsf-to-ssf-surfaces-1}}

Here, we are leveraging the fact that there are other fisher that we
never modeled to test our inferences.

\begin{Shaded}
\begin{Highlighting}[]
\NormalTok{ssf1 }\OtherTok{\textless{}{-}}\NormalTok{ amt\_fisher }\SpecialCharTok{\%\textgreater{}\%} 
  \FunctionTok{filter}\NormalTok{(name }\SpecialCharTok{==} \StringTok{"Leroy"}\NormalTok{) }\SpecialCharTok{\%\textgreater{}\%} 
      \FunctionTok{extract\_covariates}\NormalTok{(}\FunctionTok{stack}\NormalTok{(rsf.prob.raster, page.rank.raster)) }

\NormalTok{quantiles.pred }\OtherTok{\textless{}{-}} \FunctionTok{quantile}\NormalTok{(}\FunctionTok{values}\NormalTok{(rsf.prob.raster), }\FunctionTok{seq}\NormalTok{(}\FloatTok{0.01}\NormalTok{, }\DecValTok{1}\NormalTok{, }\AttributeTok{by =} \FloatTok{0.01}\NormalTok{), }\AttributeTok{na.rm =}\NormalTok{ T)}
\NormalTok{percent.in.a }\OtherTok{\textless{}{-}} \FunctionTok{rep}\NormalTok{(}\DecValTok{0}\NormalTok{,}\DecValTok{100}\NormalTok{)}
\ControlFlowTok{for}\NormalTok{ (i }\ControlFlowTok{in} \DecValTok{1}\SpecialCharTok{:}\NormalTok{(}\FunctionTok{length}\NormalTok{(quantiles.pred)))\{}
\NormalTok{  yup }\OtherTok{\textless{}{-}}\NormalTok{ ssf1}\SpecialCharTok{$}\NormalTok{layer}\FloatTok{.1} \SpecialCharTok{\textless{}=}\NormalTok{ quantiles.pred[i]}
\NormalTok{  percent.in.a[i] }\OtherTok{\textless{}{-}} \FunctionTok{sum}\NormalTok{(yup, }\AttributeTok{na.rm =}\NormalTok{ T)}
\NormalTok{\}}
\NormalTok{percent.in.a }\OtherTok{\textless{}{-}}\NormalTok{ percent.in.a}\SpecialCharTok{/}\FunctionTok{table}\NormalTok{(}\FunctionTok{is.na}\NormalTok{(ssf1}\SpecialCharTok{$}\NormalTok{layer}\FloatTok{.1}\NormalTok{))[}\DecValTok{1}\NormalTok{]}

\NormalTok{quantiles.pred }\OtherTok{\textless{}{-}} \FunctionTok{quantile}\NormalTok{(}\FunctionTok{values}\NormalTok{(page.rank.raster), }\FunctionTok{seq}\NormalTok{(}\FloatTok{0.01}\NormalTok{, }\DecValTok{1}\NormalTok{, }\AttributeTok{by =} \FloatTok{0.01}\NormalTok{), }\AttributeTok{na.rm =}\NormalTok{ T)}
\NormalTok{percent.in.b }\OtherTok{\textless{}{-}} \FunctionTok{rep}\NormalTok{(}\DecValTok{0}\NormalTok{,}\DecValTok{100}\NormalTok{)}
\ControlFlowTok{for}\NormalTok{ (i }\ControlFlowTok{in} \DecValTok{1}\SpecialCharTok{:}\NormalTok{(}\FunctionTok{length}\NormalTok{(quantiles.pred)))\{}
\NormalTok{  yup }\OtherTok{\textless{}{-}}\NormalTok{ ssf1}\SpecialCharTok{$}\NormalTok{layer}\FloatTok{.2} \SpecialCharTok{\textless{}=}\NormalTok{ quantiles.pred[i]}
\NormalTok{  percent.in.b[i] }\OtherTok{\textless{}{-}} \FunctionTok{sum}\NormalTok{(yup, }\AttributeTok{na.rm =}\NormalTok{ T)}
\NormalTok{\}}
\NormalTok{percent.in.b }\OtherTok{\textless{}{-}}\NormalTok{ percent.in.b}\SpecialCharTok{/}\FunctionTok{table}\NormalTok{(}\FunctionTok{is.na}\NormalTok{(ssf1}\SpecialCharTok{$}\NormalTok{layer}\FloatTok{.2}\NormalTok{))[}\DecValTok{1}\NormalTok{]}

\FunctionTok{tibble}\NormalTok{(}\AttributeTok{percent\_pred =} \FunctionTok{rep}\NormalTok{(}\FunctionTok{seq}\NormalTok{(}\DecValTok{0}\NormalTok{, }\DecValTok{1}\NormalTok{, }\AttributeTok{by =} \FloatTok{0.01}\NormalTok{)[}\SpecialCharTok{{-}}\DecValTok{1}\NormalTok{],}\DecValTok{2}\NormalTok{),}
       \AttributeTok{precent\_obs =} \FunctionTok{c}\NormalTok{(}\DecValTok{1}\SpecialCharTok{{-}}\NormalTok{percent.in.a,}
                       \DecValTok{1}\SpecialCharTok{{-}}\NormalTok{percent.in.b),}
       \AttributeTok{pred\_type =} \FunctionTok{rep}\NormalTok{(}\FunctionTok{c}\NormalTok{(}\StringTok{"RSF"}\NormalTok{, }\StringTok{"PageRank iSFF"}\NormalTok{), }
                       \AttributeTok{each =} \DecValTok{100}\NormalTok{)) }\SpecialCharTok{\%\textgreater{}\%} 
  \FunctionTok{group\_by}\NormalTok{(pred\_type) }\SpecialCharTok{\%\textgreater{}\%} 
  \CommentTok{\# mutate(cum\_percent = cumsum(precent\_obs)) \%\textgreater{}\% }
  \FunctionTok{ggplot}\NormalTok{(}\FunctionTok{aes}\NormalTok{(}\AttributeTok{x =}\NormalTok{ percent\_pred, }\AttributeTok{y =}\NormalTok{ precent\_obs, }\AttributeTok{color =}\NormalTok{ pred\_type)) }\SpecialCharTok{+}
  \FunctionTok{geom\_point}\NormalTok{() }\SpecialCharTok{+}
  \FunctionTok{geom\_line}\NormalTok{() }\SpecialCharTok{+}
  \FunctionTok{theme\_minimal}\NormalTok{() }\SpecialCharTok{+}
  \FunctionTok{labs}\NormalTok{(}\AttributeTok{x =} \StringTok{"Percentile of Predicted Surface"}\NormalTok{,}
       \AttributeTok{y =} \StringTok{"Cumulative Percent of Observed Data by Percentile"}\NormalTok{)}
\end{Highlighting}
\end{Shaded}

\includegraphics{Untitled_files/figure-latex/unnamed-chunk-41-1.pdf}

\begin{Shaded}
\begin{Highlighting}[]
\FunctionTok{plot}\NormalTok{(page.rank.raster}\SpecialCharTok{\textgreater{}}\NormalTok{quantiles.pred[}\FunctionTok{min}\NormalTok{(}\FunctionTok{which}\NormalTok{((}\DecValTok{1}\SpecialCharTok{{-}}\NormalTok{percent.in.b) }\SpecialCharTok{\textless{}} \FloatTok{0.9}\NormalTok{))])}
\FunctionTok{points}\NormalTok{(amt\_fisher, }\AttributeTok{pch =} \StringTok{"."}\NormalTok{, }\AttributeTok{col =} \FunctionTok{alpha}\NormalTok{(}\StringTok{"black"}\NormalTok{, }\FloatTok{0.1}\NormalTok{))}
\end{Highlighting}
\end{Shaded}

\includegraphics{Untitled_files/figure-latex/unnamed-chunk-42-1.pdf}

\hypertarget{bighorn-sheep-not-public}{%
\subsection{Bighorn Sheep (not public)}\label{bighorn-sheep-not-public}}

This is data for a 59 sheep from SW Alberta, with elevation, forest
canopy, distance to escape terrain, land cover, NDVI amplitude and
maximum, slope, and vector ruggedness. I trimmed the data set for the
movement models down to 4 individuals.

\hypertarget{data-2}{%
\subsubsection{Data}\label{data-2}}

\begin{Shaded}
\begin{Highlighting}[]
\NormalTok{files }\OtherTok{\textless{}{-}} \FunctionTok{c}\NormalTok{(}\StringTok{"aspect.tif"}\NormalTok{,}\StringTok{"canopy.tif"}\NormalTok{,}\StringTok{"descp37.tif"}\NormalTok{,}\StringTok{"elev.tif"}\NormalTok{,}\StringTok{"escp37.tif"}\NormalTok{,}\StringTok{"lc\_for.tif"}\NormalTok{,}\StringTok{"lc\_grass.tif"}\NormalTok{,}\StringTok{"lc\_other.tif"}\NormalTok{,}\StringTok{"lc\_rock.tif"}\NormalTok{,}\StringTok{"lc\_shrub.tif"}\NormalTok{,}\StringTok{"ndvi\_amp.tif"}\NormalTok{,}\StringTok{"ndvi\_max.tif"}\NormalTok{,}\StringTok{"slope.tif"}\NormalTok{,}\StringTok{"vrm3.tif"}\NormalTok{)}

\NormalTok{raster\_list }\OtherTok{\textless{}{-}} \FunctionTok{c}\NormalTok{()}
\ControlFlowTok{for}\NormalTok{ (i }\ControlFlowTok{in} \DecValTok{1}\SpecialCharTok{:}\FunctionTok{length}\NormalTok{(files)) \{}
\NormalTok{  tmp }\OtherTok{\textless{}{-}} \FunctionTok{str\_remove}\NormalTok{(files[i], }\StringTok{".tif"}\NormalTok{)}
\NormalTok{  name }\OtherTok{\textless{}{-}} \FunctionTok{paste0}\NormalTok{(}\StringTok{"\textasciitilde{}/Downloads/Research/BHS\_Mvmt/Data/RSF{-}2/GIS/writtenrasters/"}\NormalTok{,}
\NormalTok{                 files[i])}
\NormalTok{  raster\_list }\OtherTok{\textless{}{-}} \FunctionTok{append}\NormalTok{(raster\_list,  }\FunctionTok{assign}\NormalTok{(tmp, }\FunctionTok{raster}\NormalTok{(name)))}
\NormalTok{\}}
\NormalTok{rstack }\OtherTok{\textless{}{-}} \FunctionTok{stack}\NormalTok{(raster\_list)}
\NormalTok{rstack. }\OtherTok{\textless{}{-}} \FunctionTok{aggregate}\NormalTok{(rstack, }\DecValTok{50}\NormalTok{)}
\CommentTok{\# plot(rstack.)}
\FunctionTok{length}\NormalTok{(rstack.)}
\end{Highlighting}
\end{Shaded}

\begin{verbatim}
## [1] 58548
\end{verbatim}

\begin{Shaded}
\begin{Highlighting}[]
\FunctionTok{plot}\NormalTok{(}\FunctionTok{sin}\NormalTok{(pi}\SpecialCharTok{*}\NormalTok{rstack}\SpecialCharTok{$}\NormalTok{aspect))}
\end{Highlighting}
\end{Shaded}

\includegraphics{Untitled_files/figure-latex/unnamed-chunk-43-1.pdf}

The movement tracks are already in individual step formats. I also
appended some info from a HMM segmentation.

\begin{Shaded}
\begin{Highlighting}[]
\NormalTok{processed }\OtherTok{\textless{}{-}} \FunctionTok{readRDS}\NormalTok{(}\StringTok{"\textasciitilde{}/Downloads/Research/BHS\_Mvmt/Sheep\_Mvmt/Manuscript Code/Processed.rds"}\NormalTok{)}

\NormalTok{four.ind }\OtherTok{\textless{}{-}}\NormalTok{ processed }\SpecialCharTok{\%\textgreater{}\%} 
  \FunctionTok{filter}\NormalTok{(ID }\SpecialCharTok{\%in\%} \FunctionTok{c}\NormalTok{(}\StringTok{"282\_2008"}\NormalTok{, }\StringTok{"593\_2008"}\NormalTok{, }\StringTok{"600\_2007"}\NormalTok{, }\StringTok{"677\_2005"}\NormalTok{)) }\SpecialCharTok{\%\textgreater{}\%} 
  \FunctionTok{group\_by}\NormalTok{(ID) }\SpecialCharTok{\%\textgreater{}\%} 
  \FunctionTok{distinct}\NormalTok{(dt, }\AttributeTok{.keep\_all =}\NormalTok{ T) }\SpecialCharTok{\%\textgreater{}\%}
  \FunctionTok{ungroup}\NormalTok{() }\SpecialCharTok{\%\textgreater{}\%} 
  \FunctionTok{mutate}\NormalTok{(}\AttributeTok{date\_time =} \FunctionTok{as.POSIXct}\NormalTok{(date\_time)) }\SpecialCharTok{\%\textgreater{}\%} 
\NormalTok{  dplyr}\SpecialCharTok{::}\FunctionTok{select}\NormalTok{(}\AttributeTok{x =}\NormalTok{ x}\FloatTok{.1}\NormalTok{, }\AttributeTok{y =}\NormalTok{ y}\FloatTok{.1}\NormalTok{, }
         \AttributeTok{t =}\NormalTok{ date\_time, }\AttributeTok{id =}\NormalTok{ ID, }\AttributeTok{age =}\NormalTok{ age, }\AttributeTok{state =}\NormalTok{ states) }\SpecialCharTok{\%\textgreater{}\%} 
  \FunctionTok{arrange\_at}\NormalTok{(}\FunctionTok{c}\NormalTok{(}\StringTok{"id"}\NormalTok{, }\StringTok{"t"}\NormalTok{)) }\SpecialCharTok{\%\textgreater{}\%} 
  \FunctionTok{nest}\NormalTok{(}\AttributeTok{data =} \SpecialCharTok{{-}}\FunctionTok{c}\NormalTok{(}\StringTok{"id"}\NormalTok{)) }\SpecialCharTok{\%\textgreater{}\%} 
  \FunctionTok{mutate}\NormalTok{(}\AttributeTok{trk =} \FunctionTok{lapply}\NormalTok{(data,}
                      \ControlFlowTok{function}\NormalTok{(d) \{}
                        \FunctionTok{make\_track}\NormalTok{(d, x, y, t, }
                                   \AttributeTok{age =}\NormalTok{ age, }
                                   \AttributeTok{state =}\NormalTok{ state, }\CommentTok{\# this is from a HMM segmentation in three de novo states}
                                   \AttributeTok{crs =}\NormalTok{ sp}\SpecialCharTok{::}\FunctionTok{CRS}\NormalTok{(}\StringTok{"+proj=utm +zone=11"}\NormalTok{))}
\NormalTok{                        \})) }\SpecialCharTok{\%\textgreater{}\%} 
  \FunctionTok{mutate}\NormalTok{(}\AttributeTok{steps =} \FunctionTok{map}\NormalTok{(trk, }\ControlFlowTok{function}\NormalTok{(x) \{}
\NormalTok{    x }\SpecialCharTok{\%\textgreater{}\%}
      \FunctionTok{track\_resample}\NormalTok{(}\AttributeTok{rate =} \FunctionTok{minutes}\NormalTok{(}\DecValTok{30}\NormalTok{), }\AttributeTok{tolerance =} \FunctionTok{minutes}\NormalTok{(}\DecValTok{10}\NormalTok{)) }\SpecialCharTok{\%\textgreater{}\%}
      \FunctionTok{steps\_by\_burst}\NormalTok{(}\AttributeTok{keep\_cols =} \StringTok{"start"}\NormalTok{) }\SpecialCharTok{\%\textgreater{}\%}
      \FunctionTok{random\_steps}\NormalTok{(}\DecValTok{15}\NormalTok{) }\SpecialCharTok{\%\textgreater{}\%}
      \FunctionTok{extract\_covariates}\NormalTok{(rstack.) }\SpecialCharTok{\%\textgreater{}\%} 
      \FunctionTok{mutate}\NormalTok{(}
        \CommentTok{\# lc\_shrub = factor(lc\_shrub),}
        \CommentTok{\# lc\_for = factor(lc\_for),}
        \CommentTok{\# lc\_grass = factor(lc\_grass),}
        \CommentTok{\# lc\_other = factor(lc\_other),}
        \CommentTok{\# lc\_rock = factor(lc\_rock),}
        \CommentTok{\# escp37 = factor(escp37),}
        \AttributeTok{cos\_ta =} \FunctionTok{cos}\NormalTok{(ta\_),}
        \AttributeTok{log\_sl =} \FunctionTok{log}\NormalTok{(sl\_}\SpecialCharTok{+}\DecValTok{1}\NormalTok{))}
\NormalTok{  \}))}
\end{Highlighting}
\end{Shaded}

\begin{verbatim}
## Warning: 'make_track' is deprecated.
## Use 'It looks like you used `CRS()` to create the crs,
##                   please use the ESPG directly.' instead.
## See help("Deprecated")

## Warning: 'make_track' is deprecated.
## Use 'It looks like you used `CRS()` to create the crs,
##                   please use the ESPG directly.' instead.
## See help("Deprecated")

## Warning: 'make_track' is deprecated.
## Use 'It looks like you used `CRS()` to create the crs,
##                   please use the ESPG directly.' instead.
## See help("Deprecated")

## Warning: 'make_track' is deprecated.
## Use 'It looks like you used `CRS()` to create the crs,
##                   please use the ESPG directly.' instead.
## See help("Deprecated")
\end{verbatim}

\hypertarget{model-2}{%
\subsubsection{Model}\label{model-2}}

\begin{Shaded}
\begin{Highlighting}[]
\NormalTok{m2 }\OtherTok{\textless{}{-}}\NormalTok{ four.ind }\SpecialCharTok{\%\textgreater{}\%}
  \FunctionTok{mutate}\NormalTok{(}\AttributeTok{model =} \FunctionTok{map}\NormalTok{(steps, }\ControlFlowTok{function}\NormalTok{(x) \{}
\NormalTok{    x }\SpecialCharTok{\%\textgreater{}\%} 
      \FunctionTok{fit\_clogit}\NormalTok{(case\_ }\SpecialCharTok{\textasciitilde{}}\NormalTok{ log\_sl }\SpecialCharTok{+} \FunctionTok{poly}\NormalTok{(canopy,}\DecValTok{2}\NormalTok{) }\SpecialCharTok{+} \FunctionTok{poly}\NormalTok{(vrm3,}\DecValTok{2}\NormalTok{) }\SpecialCharTok{+} \FunctionTok{poly}\NormalTok{(lc\_grass,}\DecValTok{1}\NormalTok{) }\SpecialCharTok{+} \FunctionTok{poly}\NormalTok{(lc\_shrub,}\DecValTok{2}\NormalTok{) }\SpecialCharTok{+} \FunctionTok{poly}\NormalTok{(lc\_rock,}\DecValTok{2}\NormalTok{) }\SpecialCharTok{+} \FunctionTok{poly}\NormalTok{(ndvi\_amp,}\DecValTok{1}\NormalTok{) }\SpecialCharTok{+}\NormalTok{ cos\_ta }\SpecialCharTok{+} \FunctionTok{strata}\NormalTok{(step\_id\_), }\AttributeTok{model =}\NormalTok{ T)}
\NormalTok{  \}))}
\end{Highlighting}
\end{Shaded}

\hypertarget{surface-2}{%
\subsubsection{Surface}\label{surface-2}}

\begin{Shaded}
\begin{Highlighting}[]
\NormalTok{mock.surface }\OtherTok{\textless{}{-}} \FunctionTok{create\_mock\_surface}\NormalTok{(rstack., F, }\FunctionTok{list}\NormalTok{(}\AttributeTok{x =} \DecValTok{1250}\NormalTok{, }\AttributeTok{y =} \DecValTok{1250}\NormalTok{))}
\end{Highlighting}
\end{Shaded}

\hypertarget{getting-cell-data-2}{%
\subsubsection{Getting cell data}\label{getting-cell-data-2}}

Since we have multiple models fit, we can just use the first model to
get everything into shape.

\begin{Shaded}
\begin{Highlighting}[]
\NormalTok{pred.data }\OtherTok{\textless{}{-}} \FunctionTok{get\_cells}\NormalTok{(m2}\SpecialCharTok{$}\NormalTok{model[[}\DecValTok{1}\NormalTok{]], }
\NormalTok{                       mock.surface,}
\NormalTok{                       rstack., }
                       \AttributeTok{accessory.x.preds =} \FunctionTok{list}\NormalTok{(}\AttributeTok{log\_sl =} \FunctionTok{log}\NormalTok{(}\FunctionTok{step\_distance}\NormalTok{(m2}\SpecialCharTok{$}\NormalTok{model[[}\DecValTok{1}\NormalTok{]], }\FloatTok{0.5}\NormalTok{)), }
                                                \AttributeTok{cos\_ta =} \DecValTok{1}\NormalTok{))}

\NormalTok{pred.data }\OtherTok{\textless{}{-}}\NormalTok{ pred.data }\SpecialCharTok{\%\textgreater{}\%} 
  \FunctionTok{arrange}\NormalTok{(cellnr)}

\CommentTok{\# pred.data$lc\_for \textless{}{-} factor(pred.data$lc\_for)}
\CommentTok{\# pred.data$lc\_grass \textless{}{-} factor(pred.data$lc\_grass)}
\CommentTok{\# pred.data$lc\_other \textless{}{-} factor(pred.data$lc\_other)}
\CommentTok{\# pred.data$lc\_shrub \textless{}{-} factor(pred.data$lc\_shrub)}
\CommentTok{\# pred.data$lc\_rock \textless{}{-} factor(pred.data$lc\_rock)}
\CommentTok{\# pred.data$escp37 \textless{}{-} factor(pred.data$escp37)}
\end{Highlighting}
\end{Shaded}

\hypertarget{testing-surface-2}{%
\subsubsection{Testing surface}\label{testing-surface-2}}

\begin{Shaded}
\begin{Highlighting}[]
\NormalTok{cell.data }\OtherTok{\textless{}{-}} \FunctionTok{get\_cell\_data}\NormalTok{(m2}\SpecialCharTok{$}\NormalTok{model[[}\DecValTok{1}\NormalTok{]], pred.data)}
\FunctionTok{plot}\NormalTok{(}\FunctionTok{setValues}\NormalTok{(mock.surface, cell.data}\SpecialCharTok{$}\NormalTok{lRSS), }\AttributeTok{useRaster =}\NormalTok{ T, }\AttributeTok{interpolate =}\NormalTok{ F)}
\FunctionTok{lines}\NormalTok{(m2}\SpecialCharTok{$}\NormalTok{trk[[}\DecValTok{1}\NormalTok{]])}
\end{Highlighting}
\end{Shaded}

\includegraphics{Untitled_files/figure-latex/unnamed-chunk-48-1.pdf}

\hypertarget{finding-neighbors-2}{%
\subsubsection{Finding neighbors}\label{finding-neighbors-2}}

\begin{Shaded}
\begin{Highlighting}[]
\NormalTok{cell.data.list }\OtherTok{\textless{}{-}} \FunctionTok{pbmclapply}\NormalTok{(}\FunctionTok{as.list}\NormalTok{(}\DecValTok{1}\SpecialCharTok{:}\FunctionTok{dim}\NormalTok{(cell.data)[}\DecValTok{1}\NormalTok{]), }\ControlFlowTok{function}\NormalTok{(x) cell.data[x[}\DecValTok{1}\NormalTok{],])}
\NormalTok{neighbors.found }\OtherTok{\textless{}{-}} \FunctionTok{neighbor\_lookup}\NormalTok{(mock.surface, cell.data, cell.data.list)}
\end{Highlighting}
\end{Shaded}

\begin{verbatim}
## [1] "Using inputted list of cell data"
\end{verbatim}

\hypertarget{making-neighbor-comparisons-2}{%
\subsubsection{Making neighbor
comparisons}\label{making-neighbor-comparisons-2}}

\begin{Shaded}
\begin{Highlighting}[]
\NormalTok{sparse.neighbors }\OtherTok{\textless{}{-}} \FunctionTok{neighbor\_finder}\NormalTok{(m2}\SpecialCharTok{$}\NormalTok{model[[}\DecValTok{3}\NormalTok{]], cell.data, neighbors.found, }\AttributeTok{quantile =} \FloatTok{0.999999999999999}\NormalTok{, cell.data.list)}
\end{Highlighting}
\end{Shaded}

\begin{verbatim}
## [1] "Creating neighbor comparisons"
## [1] "Finding valid comparisons"
## [1] "Using inputted list of cell data"
## [1] "Running comparisons"
\end{verbatim}

\hypertarget{compiling-ssf-comparisons-2}{%
\subsubsection{Compiling SSF
comparisons}\label{compiling-ssf-comparisons-2}}

\begin{Shaded}
\begin{Highlighting}[]
\NormalTok{ssf.comparisons }\OtherTok{\textless{}{-}} \FunctionTok{compile\_ssf\_comparisons}\NormalTok{(sparse.neighbors, cell.data)}

\NormalTok{ssf.comparisons }\OtherTok{\textless{}{-}} \FunctionTok{lapply}\NormalTok{(ssf.comparisons, }\ControlFlowTok{function}\NormalTok{(x) \{}
\NormalTok{  x}\SpecialCharTok{$}\NormalTok{.for}\SpecialCharTok{$}\NormalTok{log\_sl }\OtherTok{\textless{}{-}} \FunctionTok{log}\NormalTok{(x}\SpecialCharTok{$}\NormalTok{.for}\SpecialCharTok{$}\NormalTok{step}\SpecialCharTok{+}\DecValTok{1}\NormalTok{)}
\NormalTok{  x}\SpecialCharTok{$}\NormalTok{.given}\SpecialCharTok{$}\NormalTok{log\_sl }\OtherTok{\textless{}{-}} \FunctionTok{log}\NormalTok{(x}\SpecialCharTok{$}\NormalTok{.given}\SpecialCharTok{$}\NormalTok{step}\SpecialCharTok{+}\DecValTok{1}\NormalTok{)}
  
\NormalTok{  x}\SpecialCharTok{$}\NormalTok{.for}\SpecialCharTok{$}\NormalTok{cos\_ta }\OtherTok{\textless{}{-}} \DecValTok{0}
\NormalTok{  x}\SpecialCharTok{$}\NormalTok{.given}\SpecialCharTok{$}\NormalTok{cos\_ta }\OtherTok{\textless{}{-}} \DecValTok{0}
  
  \FunctionTok{list}\NormalTok{(}\AttributeTok{.for =}\NormalTok{ x}\SpecialCharTok{$}\NormalTok{.for, }\AttributeTok{.given =}\NormalTok{ x}\SpecialCharTok{$}\NormalTok{.given)}
\NormalTok{\})}
\end{Highlighting}
\end{Shaded}

\hypertarget{predicting-surface-probabilities-2}{%
\subsubsection{Predicting surface
probabilities}\label{predicting-surface-probabilities-2}}

TIDYVERSE BABYYYYYY

\begin{Shaded}
\begin{Highlighting}[]
\NormalTok{predicted.surfaces }\OtherTok{\textless{}{-}}\NormalTok{ m2 }\SpecialCharTok{\%\textgreater{}\%} 
  \FunctionTok{mutate}\NormalTok{(}\AttributeTok{surfaces =} \FunctionTok{map}\NormalTok{(model, }\ControlFlowTok{function}\NormalTok{(x) \{}
    \FunctionTok{predict\_ssf\_comparisons}\NormalTok{(x, ssf.comparisons)}
\NormalTok{  \}))}
\end{Highlighting}
\end{Shaded}

\begin{verbatim}
## [1] "Estimating probability surface"
## [1] "Compiling probability surface"
## [1] "Making sparse matrix for transitions"
## [1] "Estimating probability surface"
## [1] "Compiling probability surface"
## [1] "Making sparse matrix for transitions"
## [1] "Estimating probability surface"
## [1] "Compiling probability surface"
## [1] "Making sparse matrix for transitions"
## [1] "Estimating probability surface"
## [1] "Compiling probability surface"
## [1] "Making sparse matrix for transitions"
\end{verbatim}

\hypertarget{making-graph-2}{%
\subsubsection{Making graph}\label{making-graph-2}}

TIDYVERSE AGAIN BABYYYYYY \#roUND 2

\begin{Shaded}
\begin{Highlighting}[]
\NormalTok{graphs }\OtherTok{\textless{}{-}}\NormalTok{ predicted.surfaces }\SpecialCharTok{\%\textgreater{}\%} 
  \FunctionTok{mutate}\NormalTok{(}\AttributeTok{graph =} \FunctionTok{map}\NormalTok{(surfaces, }\ControlFlowTok{function}\NormalTok{(x) \{}
\NormalTok{    g }\OtherTok{\textless{}{-}} \FunctionTok{graph\_from\_adjacency\_matrix}\NormalTok{(x}\SpecialCharTok{$}\NormalTok{sparse.matrix, }\AttributeTok{mode =} \StringTok{"directed"}\NormalTok{, }\AttributeTok{weighted =}\NormalTok{ T, }\AttributeTok{diag =}\NormalTok{ T)}
    \FunctionTok{V}\NormalTok{(g)}\SpecialCharTok{$}\NormalTok{name }\OtherTok{\textless{}{-}} \FunctionTok{V}\NormalTok{(g)}
\NormalTok{    Isolated }\OtherTok{\textless{}{-}} \FunctionTok{which}\NormalTok{(}\FunctionTok{degree}\NormalTok{(g)}\SpecialCharTok{==}\DecValTok{0}\NormalTok{)}
\NormalTok{    Connected }\OtherTok{\textless{}{-}} \FunctionTok{which}\NormalTok{(}\FunctionTok{degree}\NormalTok{(g)}\SpecialCharTok{\textgreater{}}\DecValTok{0}\NormalTok{)}
\NormalTok{    g }\OtherTok{\textless{}{-}} \FunctionTok{delete.vertices}\NormalTok{(g, Isolated)}
\NormalTok{    g }\OtherTok{\textless{}{-}} \FunctionTok{delete.edges}\NormalTok{(g, }\FunctionTok{E}\NormalTok{(g)[}\FunctionTok{is.na}\NormalTok{(weight)])}
\NormalTok{    g}
\NormalTok{  \}))}
\end{Highlighting}
\end{Shaded}

\hypertarget{calculating-centrality-2}{%
\subsubsection{Calculating centrality}\label{calculating-centrality-2}}

\begin{Shaded}
\begin{Highlighting}[]
\NormalTok{centrality.surface }\OtherTok{\textless{}{-}}\NormalTok{ graphs }\SpecialCharTok{\%\textgreater{}\%} 
  \FunctionTok{mutate}\NormalTok{(}\AttributeTok{centrality =} \FunctionTok{map}\NormalTok{(graph, }\ControlFlowTok{function}\NormalTok{(x) \{}
\NormalTok{    pg }\OtherTok{\textless{}{-}} \FunctionTok{page\_rank}\NormalTok{(x)}
\NormalTok{    values }\OtherTok{\textless{}{-}} \FunctionTok{rep}\NormalTok{(}\ConstantTok{NA}\NormalTok{, }\FunctionTok{length}\NormalTok{(mock.surface))}
\NormalTok{    values[}\FunctionTok{V}\NormalTok{(x)}\SpecialCharTok{$}\NormalTok{name] }\OtherTok{\textless{}{-}}\NormalTok{ pg}\SpecialCharTok{$}\NormalTok{vector}
\NormalTok{    page.rank.raster }\OtherTok{\textless{}{-}} \FunctionTok{setValues}\NormalTok{(mock.surface, values)}
\NormalTok{    page.rank.raster}
\NormalTok{  \}))}

\FunctionTok{par}\NormalTok{(}\AttributeTok{mfrow =} \FunctionTok{c}\NormalTok{(}\DecValTok{2}\NormalTok{,}\DecValTok{2}\NormalTok{))}
\FunctionTok{plot}\NormalTok{(centrality.surface}\SpecialCharTok{$}\NormalTok{centrality[[}\DecValTok{1}\NormalTok{]]}\SpecialCharTok{\^{}}\NormalTok{(}\DecValTok{1}\SpecialCharTok{/}\DecValTok{2}\NormalTok{), }\AttributeTok{interpolate =}\NormalTok{ F)}
\FunctionTok{points}\NormalTok{(centrality.surface}\SpecialCharTok{$}\NormalTok{data[[}\DecValTok{1}\NormalTok{]], }\AttributeTok{pch =} \StringTok{"."}\NormalTok{, }\AttributeTok{col =} \FunctionTok{alpha}\NormalTok{(}\StringTok{"black"}\NormalTok{, }\FloatTok{0.1}\NormalTok{))}
\FunctionTok{plot}\NormalTok{(centrality.surface}\SpecialCharTok{$}\NormalTok{centrality[[}\DecValTok{2}\NormalTok{]], }\AttributeTok{interpolate =}\NormalTok{ F)}
\FunctionTok{points}\NormalTok{(centrality.surface}\SpecialCharTok{$}\NormalTok{data[[}\DecValTok{2}\NormalTok{]], }\AttributeTok{pch =} \StringTok{"."}\NormalTok{, }\AttributeTok{col =} \FunctionTok{alpha}\NormalTok{(}\StringTok{"black"}\NormalTok{, }\FloatTok{0.1}\NormalTok{))}
\FunctionTok{plot}\NormalTok{(centrality.surface}\SpecialCharTok{$}\NormalTok{centrality[[}\DecValTok{3}\NormalTok{]], }\AttributeTok{interpolate =}\NormalTok{ F)}
\FunctionTok{points}\NormalTok{(centrality.surface}\SpecialCharTok{$}\NormalTok{data[[}\DecValTok{3}\NormalTok{]], }\AttributeTok{pch =} \StringTok{"."}\NormalTok{, }\AttributeTok{col =} \FunctionTok{alpha}\NormalTok{(}\StringTok{"black"}\NormalTok{, }\FloatTok{0.1}\NormalTok{))}
\FunctionTok{plot}\NormalTok{(centrality.surface}\SpecialCharTok{$}\NormalTok{centrality[[}\DecValTok{4}\NormalTok{]], }\AttributeTok{interpolate =}\NormalTok{ F)}
\FunctionTok{points}\NormalTok{(centrality.surface}\SpecialCharTok{$}\NormalTok{data[[}\DecValTok{4}\NormalTok{]], }\AttributeTok{pch =} \StringTok{"."}\NormalTok{, }\AttributeTok{col =} \FunctionTok{alpha}\NormalTok{(}\StringTok{"black"}\NormalTok{, }\FloatTok{0.1}\NormalTok{))}
\end{Highlighting}
\end{Shaded}

\includegraphics{Untitled_files/figure-latex/unnamed-chunk-54-1.pdf}

\begin{Shaded}
\begin{Highlighting}[]
\FunctionTok{cor}\NormalTok{(}\FunctionTok{cbind}\NormalTok{(}\FunctionTok{values}\NormalTok{(centrality.surface}\SpecialCharTok{$}\NormalTok{centrality[[}\DecValTok{1}\NormalTok{]]), }
          \FunctionTok{values}\NormalTok{(centrality.surface}\SpecialCharTok{$}\NormalTok{centrality[[}\DecValTok{2}\NormalTok{]]),}
          \FunctionTok{values}\NormalTok{(centrality.surface}\SpecialCharTok{$}\NormalTok{centrality[[}\DecValTok{3}\NormalTok{]]),}
          \FunctionTok{values}\NormalTok{(centrality.surface}\SpecialCharTok{$}\NormalTok{centrality[[}\DecValTok{4}\NormalTok{]])),}
    \AttributeTok{use =} \StringTok{"pairwise.complete"}\NormalTok{,}
    \AttributeTok{method =} \StringTok{"spearman"}\NormalTok{)}
\end{Highlighting}
\end{Shaded}

\begin{verbatim}
##           [,1]      [,2]      [,3]      [,4]
## [1,] 1.0000000 0.3559888 0.5015261 0.7121948
## [2,] 0.3559888 1.0000000 0.7852601 0.3712916
## [3,] 0.5015261 0.7852601 1.0000000 0.4463302
## [4,] 0.7121948 0.3712916 0.4463302 1.0000000
\end{verbatim}

\hypertarget{comparing-rsf-to-ssf-surfaces-2}{%
\subsubsection{Comparing RSF to SSF
surfaces}\label{comparing-rsf-to-ssf-surfaces-2}}

Here, we are leveraging the fact that there are other fisher that we
never modeled to test our inferences.

\begin{Shaded}
\begin{Highlighting}[]
\NormalTok{comparison }\OtherTok{\textless{}{-}}\NormalTok{ processed}

\NormalTok{extracted }\OtherTok{\textless{}{-}}\NormalTok{ raster}\SpecialCharTok{::}\FunctionTok{extract}\NormalTok{(}\FunctionTok{stack}\NormalTok{(centrality.surface}\SpecialCharTok{$}\NormalTok{centrality), }\FunctionTok{as.matrix}\NormalTok{(comparison[,}\FunctionTok{c}\NormalTok{(}\StringTok{"x.1"}\NormalTok{, }\StringTok{"y.1"}\NormalTok{)]), }\AttributeTok{df=}\ConstantTok{TRUE}\NormalTok{)}
  
\NormalTok{quantiles }\OtherTok{\textless{}{-}} \FunctionTok{lapply}\NormalTok{(centrality.surface}\SpecialCharTok{$}\NormalTok{centrality, }\ControlFlowTok{function}\NormalTok{(x)\{}
    \FunctionTok{quantile}\NormalTok{(}\FunctionTok{values}\NormalTok{(x), }\FunctionTok{seq}\NormalTok{(}\FloatTok{0.01}\NormalTok{, }\DecValTok{1}\NormalTok{, }\AttributeTok{by =} \FloatTok{0.01}\NormalTok{), }\AttributeTok{na.rm =}\NormalTok{ T)}
\NormalTok{  \})}

\NormalTok{percent.in}\FloatTok{.1} \OtherTok{\textless{}{-}} \FunctionTok{rep}\NormalTok{(}\DecValTok{0}\NormalTok{,}\DecValTok{100}\NormalTok{)}
\ControlFlowTok{for}\NormalTok{ (i }\ControlFlowTok{in} \DecValTok{1}\SpecialCharTok{:}\NormalTok{(}\FunctionTok{length}\NormalTok{(percent.in}\FloatTok{.1}\NormalTok{)))\{}
\NormalTok{  yup }\OtherTok{\textless{}{-}}\NormalTok{ extracted}\SpecialCharTok{$}\NormalTok{layer}\FloatTok{.1} \SpecialCharTok{\textless{}=}\NormalTok{ quantiles[[}\DecValTok{1}\NormalTok{]][i]}
\NormalTok{  percent.in}\FloatTok{.1}\NormalTok{[i] }\OtherTok{\textless{}{-}} \FunctionTok{sum}\NormalTok{(yup, }\AttributeTok{na.rm =}\NormalTok{ T)}
\NormalTok{\}}
\NormalTok{percent.in}\FloatTok{.1} \OtherTok{\textless{}{-}}\NormalTok{ percent.in}\FloatTok{.1}\SpecialCharTok{/}\FunctionTok{table}\NormalTok{(}\FunctionTok{is.na}\NormalTok{(extracted}\SpecialCharTok{$}\NormalTok{layer}\FloatTok{.1}\NormalTok{))[}\DecValTok{1}\NormalTok{]}

\NormalTok{percent.in}\FloatTok{.2} \OtherTok{\textless{}{-}} \FunctionTok{rep}\NormalTok{(}\DecValTok{0}\NormalTok{,}\DecValTok{100}\NormalTok{)}
\ControlFlowTok{for}\NormalTok{ (i }\ControlFlowTok{in} \DecValTok{1}\SpecialCharTok{:}\NormalTok{(}\FunctionTok{length}\NormalTok{(percent.in}\FloatTok{.2}\NormalTok{)))\{}
\NormalTok{  yup }\OtherTok{\textless{}{-}}\NormalTok{ extracted}\SpecialCharTok{$}\NormalTok{layer}\FloatTok{.2} \SpecialCharTok{\textless{}=}\NormalTok{ quantiles[[}\DecValTok{1}\NormalTok{]][i]}
\NormalTok{  percent.in}\FloatTok{.2}\NormalTok{[i] }\OtherTok{\textless{}{-}} \FunctionTok{sum}\NormalTok{(yup, }\AttributeTok{na.rm =}\NormalTok{ T)}
\NormalTok{\}}
\NormalTok{percent.in}\FloatTok{.2} \OtherTok{\textless{}{-}}\NormalTok{ percent.in}\FloatTok{.2}\SpecialCharTok{/}\FunctionTok{table}\NormalTok{(}\FunctionTok{is.na}\NormalTok{(extracted}\SpecialCharTok{$}\NormalTok{layer}\FloatTok{.2}\NormalTok{))[}\DecValTok{1}\NormalTok{]}

\NormalTok{percent.in}\FloatTok{.3} \OtherTok{\textless{}{-}} \FunctionTok{rep}\NormalTok{(}\DecValTok{0}\NormalTok{,}\DecValTok{100}\NormalTok{)}
\ControlFlowTok{for}\NormalTok{ (i }\ControlFlowTok{in} \DecValTok{1}\SpecialCharTok{:}\NormalTok{(}\FunctionTok{length}\NormalTok{(percent.in}\FloatTok{.3}\NormalTok{)))\{}
\NormalTok{  yup }\OtherTok{\textless{}{-}}\NormalTok{ extracted}\SpecialCharTok{$}\NormalTok{layer}\FloatTok{.3} \SpecialCharTok{\textless{}=}\NormalTok{ quantiles[[}\DecValTok{1}\NormalTok{]][i]}
\NormalTok{  percent.in}\FloatTok{.3}\NormalTok{[i] }\OtherTok{\textless{}{-}} \FunctionTok{sum}\NormalTok{(yup, }\AttributeTok{na.rm =}\NormalTok{ T)}
\NormalTok{\}}
\NormalTok{percent.in}\FloatTok{.3} \OtherTok{\textless{}{-}}\NormalTok{ percent.in}\FloatTok{.3}\SpecialCharTok{/}\FunctionTok{table}\NormalTok{(}\FunctionTok{is.na}\NormalTok{(extracted}\SpecialCharTok{$}\NormalTok{layer}\FloatTok{.3}\NormalTok{))[}\DecValTok{1}\NormalTok{]}

\NormalTok{percent.in}\FloatTok{.4} \OtherTok{\textless{}{-}} \FunctionTok{rep}\NormalTok{(}\DecValTok{0}\NormalTok{,}\DecValTok{100}\NormalTok{)}
\ControlFlowTok{for}\NormalTok{ (i }\ControlFlowTok{in} \DecValTok{1}\SpecialCharTok{:}\NormalTok{(}\FunctionTok{length}\NormalTok{(percent.in}\FloatTok{.4}\NormalTok{)))\{}
\NormalTok{  yup }\OtherTok{\textless{}{-}}\NormalTok{ extracted}\SpecialCharTok{$}\NormalTok{layer}\FloatTok{.4} \SpecialCharTok{\textless{}=}\NormalTok{ quantiles[[}\DecValTok{1}\NormalTok{]][i]}
\NormalTok{  percent.in}\FloatTok{.4}\NormalTok{[i] }\OtherTok{\textless{}{-}} \FunctionTok{sum}\NormalTok{(yup, }\AttributeTok{na.rm =}\NormalTok{ T)}
\NormalTok{\}}
\NormalTok{percent.in}\FloatTok{.4} \OtherTok{\textless{}{-}}\NormalTok{ percent.in}\FloatTok{.4}\SpecialCharTok{/}\FunctionTok{table}\NormalTok{(}\FunctionTok{is.na}\NormalTok{(extracted}\SpecialCharTok{$}\NormalTok{layer}\FloatTok{.4}\NormalTok{))[}\DecValTok{1}\NormalTok{]}

\FunctionTok{tibble}\NormalTok{(}\AttributeTok{percent\_pred =} \FunctionTok{rep}\NormalTok{(}\FunctionTok{seq}\NormalTok{(}\DecValTok{0}\NormalTok{, }\DecValTok{1}\NormalTok{, }\AttributeTok{by =} \FloatTok{0.01}\NormalTok{)[}\SpecialCharTok{{-}}\DecValTok{1}\NormalTok{],}\DecValTok{4}\NormalTok{),}
       \AttributeTok{precent\_obs =} 
         \CommentTok{\# c(NA,diff(percent.in.1),}
         \CommentTok{\#   NA,diff(percent.in.2),}
         \CommentTok{\#   NA,diff(percent.in.3),}
         \CommentTok{\#   NA,diff(percent.in.4)),}
         \FunctionTok{c}\NormalTok{(}\DecValTok{1}\SpecialCharTok{{-}}\NormalTok{percent.in}\FloatTok{.1}\NormalTok{,}
           \DecValTok{1}\SpecialCharTok{{-}}\NormalTok{percent.in}\FloatTok{.2}\NormalTok{,}
           \DecValTok{1}\SpecialCharTok{{-}}\NormalTok{percent.in}\FloatTok{.3}\NormalTok{,}
           \DecValTok{1}\SpecialCharTok{{-}}\NormalTok{percent.in}\FloatTok{.4}\NormalTok{),}
       \AttributeTok{pred\_type =} \FunctionTok{rep}\NormalTok{(}\FunctionTok{paste}\NormalTok{(}\StringTok{"Animal"}\NormalTok{, }\DecValTok{1}\SpecialCharTok{:}\DecValTok{4}\NormalTok{), }
                       \AttributeTok{each =} \DecValTok{100}\NormalTok{)) }\SpecialCharTok{\%\textgreater{}\%} 
  \FunctionTok{group\_by}\NormalTok{(pred\_type) }\SpecialCharTok{\%\textgreater{}\%} 
  \CommentTok{\# mutate(cum\_percent = cumsum(precent\_obs)) \%\textgreater{}\% }
  \FunctionTok{ggplot}\NormalTok{(}\FunctionTok{aes}\NormalTok{(}\AttributeTok{x =}\NormalTok{ percent\_pred, }\AttributeTok{y =}\NormalTok{ precent\_obs, }\AttributeTok{color =}\NormalTok{ pred\_type)) }\SpecialCharTok{+}
  \FunctionTok{geom\_point}\NormalTok{() }\SpecialCharTok{+}
  \FunctionTok{geom\_line}\NormalTok{() }\SpecialCharTok{+}
  \FunctionTok{theme\_minimal}\NormalTok{() }\SpecialCharTok{+}
  \FunctionTok{labs}\NormalTok{(}\AttributeTok{x =} \StringTok{"Percentile of Predicted Surface"}\NormalTok{,}
       \AttributeTok{y =} \StringTok{"Cumulative Percent of Observed Data by Percentile"}\NormalTok{)}
\end{Highlighting}
\end{Shaded}

\includegraphics{Untitled_files/figure-latex/unnamed-chunk-55-1.pdf}

Variation in animal predictions!!! Super cooooooool

\begin{Shaded}
\begin{Highlighting}[]
\NormalTok{pheno }\OtherTok{\textless{}{-}}\NormalTok{ processed[processed}\SpecialCharTok{$}\NormalTok{ID }\SpecialCharTok{\%in\%}\NormalTok{ four.ind}\SpecialCharTok{$}\NormalTok{id,]}
\FunctionTok{as.data.frame}\NormalTok{(pheno) }\SpecialCharTok{\%\textgreater{}\%} 
  \FunctionTok{distinct}\NormalTok{(ID,sex, age) }
\end{Highlighting}
\end{Shaded}

\begin{verbatim}
##         ID sex   age
## 1 282_2008   F    11
## 2 593_2008   F Adult
## 3 600_2007   F     2
## 4 677_2005   M     5
\end{verbatim}

\begin{Shaded}
\begin{Highlighting}[]
\FunctionTok{par}\NormalTok{(}\AttributeTok{mfrow =} \FunctionTok{c}\NormalTok{(}\DecValTok{2}\NormalTok{,}\DecValTok{2}\NormalTok{))}
\FunctionTok{plot}\NormalTok{(centrality.surface}\SpecialCharTok{$}\NormalTok{centrality[[}\DecValTok{1}\NormalTok{]]}\SpecialCharTok{\textgreater{}}\NormalTok{quantiles[[}\DecValTok{1}\NormalTok{]][}\FunctionTok{min}\NormalTok{(}\FunctionTok{which}\NormalTok{((}\DecValTok{1}\SpecialCharTok{{-}}\NormalTok{percent.in}\FloatTok{.1}\NormalTok{) }\SpecialCharTok{\textless{}} \FloatTok{0.5}\NormalTok{))])}
\FunctionTok{points}\NormalTok{(comparison}\SpecialCharTok{$}\NormalTok{x\_, comparison}\SpecialCharTok{$}\NormalTok{y\_, }\AttributeTok{pch =} \StringTok{"."}\NormalTok{, }\AttributeTok{col =} \FunctionTok{alpha}\NormalTok{(}\StringTok{"black"}\NormalTok{, }\AttributeTok{alpha =} \DecValTok{1}\NormalTok{))}
\FunctionTok{plot}\NormalTok{(centrality.surface}\SpecialCharTok{$}\NormalTok{centrality[[}\DecValTok{2}\NormalTok{]]}\SpecialCharTok{\textgreater{}}\NormalTok{quantiles[[}\DecValTok{2}\NormalTok{]][}\FunctionTok{min}\NormalTok{(}\FunctionTok{which}\NormalTok{((}\DecValTok{1}\SpecialCharTok{{-}}\NormalTok{percent.in}\FloatTok{.2}\NormalTok{) }\SpecialCharTok{\textless{}} \FloatTok{0.5}\NormalTok{))])}
\FunctionTok{points}\NormalTok{(comparison}\SpecialCharTok{$}\NormalTok{x\_, comparison}\SpecialCharTok{$}\NormalTok{y\_, }\AttributeTok{pch =} \StringTok{"."}\NormalTok{, }\AttributeTok{col =} \FunctionTok{alpha}\NormalTok{(}\StringTok{"black"}\NormalTok{, }\AttributeTok{alpha =} \DecValTok{1}\NormalTok{))}
\FunctionTok{plot}\NormalTok{(centrality.surface}\SpecialCharTok{$}\NormalTok{centrality[[}\DecValTok{3}\NormalTok{]]}\SpecialCharTok{\textgreater{}}\NormalTok{quantiles[[}\DecValTok{3}\NormalTok{]][}\FunctionTok{min}\NormalTok{(}\FunctionTok{which}\NormalTok{((}\DecValTok{1}\SpecialCharTok{{-}}\NormalTok{percent.in}\FloatTok{.3}\NormalTok{) }\SpecialCharTok{\textless{}} \FloatTok{0.5}\NormalTok{))])}
\FunctionTok{points}\NormalTok{(comparison}\SpecialCharTok{$}\NormalTok{x\_, comparison}\SpecialCharTok{$}\NormalTok{y\_, }\AttributeTok{pch =} \StringTok{"."}\NormalTok{, }\AttributeTok{col =} \FunctionTok{alpha}\NormalTok{(}\StringTok{"black"}\NormalTok{, }\AttributeTok{alpha =} \DecValTok{1}\NormalTok{))}
\FunctionTok{plot}\NormalTok{(centrality.surface}\SpecialCharTok{$}\NormalTok{centrality[[}\DecValTok{4}\NormalTok{]]}\SpecialCharTok{\textgreater{}}\NormalTok{quantiles[[}\DecValTok{4}\NormalTok{]][}\FunctionTok{min}\NormalTok{(}\FunctionTok{which}\NormalTok{((}\DecValTok{1}\SpecialCharTok{{-}}\NormalTok{percent.in}\FloatTok{.4}\NormalTok{) }\SpecialCharTok{\textless{}} \FloatTok{0.5}\NormalTok{))])}
\FunctionTok{points}\NormalTok{(comparison}\SpecialCharTok{$}\NormalTok{x\_, comparison}\SpecialCharTok{$}\NormalTok{y\_, }\AttributeTok{pch =} \StringTok{"."}\NormalTok{, }\AttributeTok{col =} \FunctionTok{alpha}\NormalTok{(}\StringTok{"black"}\NormalTok{, }\AttributeTok{alpha =} \DecValTok{1}\NormalTok{))}
\end{Highlighting}
\end{Shaded}

\includegraphics{Untitled_files/figure-latex/unnamed-chunk-56-1.pdf}

\end{document}
